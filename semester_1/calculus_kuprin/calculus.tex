\documentclass[a4paper]{article}

\usepackage[12pt]{extsizes}
\usepackage[left=2.5cm,right=2.5cm,top=2.5cm,bottom=3cm]{geometry}

\usepackage{hyperref}
\usepackage{xcolor}
\definecolor{linkcolor}{HTML}{799B03} % цвет ссылок
\definecolor{urlcolor}{HTML}{799B03} % цвет гиперссылок

\usepackage[russian]{babel}
\usepackage{amsmath}
\usepackage{amssymb}
\usepackage{amsfonts}

\title{Лекции по анализу}
\date{31.10.2021}
\author{Куприн А.}

\newtheorem{defin}{Определение}
\newtheorem{example}{Пример}
\newtheorem{zam}{Замечание}
\newtheorem{theor}{Теорема}


\begin{document}
\maketitle
\tableofcontents
\newpage
\begin{abstract}
Данные записки основаны на лекциях, прочитанных А. В. Куприным в первом
семестре 2021-2022 учебного года. Упор в них делается на теорию и логическую
структуру анализа. Примеры даются лишь в том случае, когда без них невозможно
объяснить смысл нового понятия. Остальные примеры можно по желанию можно найти
в источниках (разделы этих записок сопровождаются необходимыми ссылками)
\end{abstract}
\section{Множества и логические символы}
Для математики, одно из важнейших понятий - понятие множества и элемента. Они
оба входят в единственный двухместный предикат Теории Множеств - отношение
принадлежности. Когда пишут $a\in b$, это означает, в частности, что а -
элемент, b - множество.
\subsection{Логическая символика}
Мы употребляем символ $\And$ для конъюнкции, $\Rightarrow$ дял импликации, 
$\Leftrightarrow$ для эквивалентности , также используем стандартные 
обозначения кванторов: $\exists$ для квантора существования, $\forall$ для 
квантора всеобщности, $\exists!$ для обоначения существования и единственности. 

Следуя \cite{z}, мы обозначаем определение как $a:=b$, где а - определяемый
объект, b - определение
\subsection{Функции}
\begin{defin}
Декартовым произведением множеств А и В называется множество упорядоченных пар 
$A\times B=M:=\{(a,b)\mid a\in A, b\in B\}$
\end{defin}
\begin{defin}
Подмножество $f$ декартового произведения $M$ множеств $A$ и $B$ называется 
функцией из $A$ в $B$ и обозначается $f\colon A\to B$, если для всякого
элемента а из А существует не более одной упорядоченной пары (а,b), входящей 
в f:
\end{defin}
$$f\colon A\to B \Leftrightarrow \forall a\in A (\exists!h\in M\colon(a\in h)
\And(h\in f))\vee\lnot(\exists h\in M\colon(a\in h)\And(h\in f))$$
Иначе говоря, для элемента прообраза функции либо не существует образа, либо 
образ существует и единственен
\subsection{Натуральные числа}
Обычно, для определения множества натуральных чисел используется функция 
следования $x\mapsto x'$ (инкремент) и использующие её аксиомы Пеано. Для 
наших целей наиболее важна \textbf{аксиома индукции}:
$$R(1)\And(\forall n\colon R(n)\Rightarrow R(n'))\Rightarrow(\forall n\colon
R(n))$$
Иначе говоря, если некоторое свойство выполняется для единицы и из выполнения
для числа $n$ следует выполнение для числа $n'=n+1$, то это свойство выполнено
для всех натуральных чисел. 

Утверждение $R(1)$ обычно называют \textbf{базой индукции}, а $\forall n\colon
R(n)\Rightarrow R(n')$ - \textbf{индуктивным переходом}.
\section{Элементы дискретной математики}
\subsection{Бином Ньютона}

\begin{defin}
Число $C_n^k:=\frac{n!}{k!(n-k)!}$ называется биномиальным коээфициентом.
\end{defin}

\begin{theor}
$C^{k-1}_n+C^{k}_n=C^k_{n+1}$
\end{theor}
\textbf{Доказательство.} $C^{k-1}_n+C^{k}_n=\frac{n!}{(k-1)!(n-k+1)!}+\frac{n!}
{k!(n-k)!}=\frac{n!}{(k-1)!(n-k)!(n-k+1)}+\frac{n!}{(k-1)!(n-k)!k}=
\frac{n!(k+n-k+1)}{(k-1)!k(n-k)!(n-k+1)}=\frac{(n+1)!}{k!(n+1-k)!}=C^k_{n+1}$ 
$\square$

\begin{theor}
$(a+b)^n=\sum\limits^n_{k=0}C^k_na^{n-k}b{k}$ (Формула бинома Ньютона)
\end{theor}
\textbf{Доказательство.} По индукции. База: для $n=1:$ $(a+b)^1=a+b$. Переход:
$(a+b)^n(a+b)=(a+b)\sum\limits^n_{k=0}=C^k_na^{n-k}b{k}=a\sum\limits^n_{k=0}=
C^k_na^{n-k}b{k}+b\sum\limits^n_{k=0}=C^k_na^{n-k}b{k}$  $\square$


\section{Действительные числа}
\subsection{Определение и основные классы}
Выпишем аксиомы, задающее множество действительных чисел, согласно \cite{z}.
\begin{defin}
Множество $\mathbb R$ называется множеством действительных чисел, если 
удовлетворяет следующим аксиомам:
\end{defin}
1-4. $\mathbb R$ - аддитивная абелева группа\\
4-8. $\mathbb R$ - мультипликатиная абелева группа\\
9. $a(b+c)=ab+ac$ (связь сложения и умножения)\\
10.\\
11.\\
12.\\
13.\\
14.\\
15.\\
16.$\forall A,B\subset\mathbb R\colon \forall a\in A, b\in B\colon 
a>b\Rightarrow\exists c\colon a>c>b$ - аксиома полноты; для любых двух 
множеств, каждый элемент одного из которого строго больше каждого элемента 
другого, существует число c, лежащее между этими множествами \\

\subsection{Представление в десятичной записи}

\subsection{Ограниченные множества}
\subsection{Модуль числа}
\begin{theor}
Любое ограниченное множество имеет точную нижнюю (верхнюю) грань.
\end{theor}
\textbf{Доказательство} по индукции$\square$


\section{Предел и непрерывность}
\subsection{Предел последовательности}
\begin{theor}
У последовательности имеется только один предел
\end{theor}
\textbf{Доказательство} от противного$\square$
\begin{theor}
Чтобы последовательность сходилась, необходимо, чтобы она была ограниченна
\end{theor}
\textbf{Доказательство} $\square$
\begin{theor}
Если последовательность сходится, то любая её подпоследовательность сходится 
к её пределу.
\end{theor}
\textbf{Доказательство} по индукции$\square$
\begin{theor}
(лемма о вложенных отрезках/принцип Коши-Кантора)\\
Последовательность вложенных отрезков содержит точку, общую для всех отрезков. 
Если, кроме того, длина отрезков убывает, то такая точка единственна.
\end{theor}
\textbf{Доказательство} по индукции$\square$
\begin{theor}
(об ограниченной последовательности)\\
Если последовательность возрастает и ограниченна сверху, то её предел равен её 
супремуму.
\end{theor}
\textbf{Доказательство} по индукции$\square$
\begin{theor}
(о пределе  неравенствах)\\
Если имеет место $\lim\limits_{n\to\infty}x_n=a$,
$\lim\limits_{n\to\infty}y_n=b$, то $\forall n:~x_n>y_n\Rightarrow a>=b$
\end{theor}
\textbf{Доказательство} по индукции$\square$
\begin{theor}
(Больцано-Вейерштрасса)\\
Из всякой ограниченной последовательности можно выделить сходящуюся 
подпоследовательность
\end{theor}
\textbf{Доказательство} 1. Если множество значений конечно. 2. В другом случае:
делим отрезок пополам $\square$
\begin{theor}
(лемма)\\
Всякая сходящаяся последовательность представляется в виде суммы предела и 
бесконечно малой последовательности. 
\end{theor}
\textbf{Доказательство} $\square$
\begin{theor}
(лемма о двух милиционерах)\\

\end{theor}
\textbf{Доказательство} $\square$
\subsection{Число е}
\begin{theor}
Существует конечный предел последовательности
$$e{:=}\lim\limits_{n\to\infty}\left(1+\frac{1}{n}\right)^n$$
\end{theor}
\textbf{Доказательство} $\square$
\subsection{Предел отношения многочленов}
\subsection{Порядок бесконечно малых}
\subsection{Примеры}



\subsection{Предел функции}

\begin{defin}
Определение по Коши
\end{defin}
\begin{defin}
Определение по Гейне
\end{defin}
\begin{theor}
Определения по Коши и Гейне эквивалентны
\end{theor}
\textbf{Доказательство} $\square$
\begin{theor}
(первый замечательный предел)\\
$\lim\limits_{x\to0}\frac{\sin x}{x}=1$
\end{theor}
\textbf{Доказательство} $\square$


\subsection{Непрерывность функции}
\begin{defin}
$f(x)$ $x_0$ $\lim\limits_{x\to x_0}f(x)=f(x_0)$
\end{defin}
\subsection{Точки разрыва}
Первого рода - если существуют с обеих сторон конечные пределы функции.\\
Второго рода - в любом другом случае. 
\begin{theor}
Если функция непрерывна на отрезке и имеет разные знаки на его концах, то 
существует точка отрезка, в которой она обращается в ноль.
\end{theor}
\textbf{Доказательство} Через принцип вложенных отрезков, разбивая исзодный и 
смотря там где сумма знакопеременна $\square$
\begin{theor} (коши о промежуточном значении)\\
Пусть функция непрерывна на отрезке. Тогда для любой точки на области значений
функции по отрезку найдется прообраз (точка с).
\end{theor}
\textbf{Доказательство} Рассмотрим функцию ф-С, где С лежит между ф(а) и ф(б).
Она алсо непрерывна и знакопеременна, сл-но, имеет ноль  $\square$
\begin{theor}
(вейерштрасса)\\
Функция непрерывна на отрезке => она ограничена на нем и достигает 
экстремальных значений. 
\end{theor}
\textbf{Доказательство} 1. пусть она не ограничена. Зафиксируем 
последовательность с пределом в бесконечности. По ТБВ выделим 
посдпоследовательность, стремящуюся к бесконечности в какой то точке отрезка. 
По гейне, предел функции этой последовательности равен бесконечности, и это
достигается в отрезке, что противоречит непрерывности.\\
2. Пусть грань не достигается. Рассмотрим функциию фи=1/(супремум-ф(х)). Она
ограничена по пункту 1, но тогда противоречие с определением верхней грани.
$\square$
\subsection{Асимптоты}



\section{Дифференцирование и производная}
\subsection{Основные определения}
\begin{defin}
Производная функции = лим ф(х+аш)-ф(х) делить на аш
\end{defin}
\begin{defin}
Функция является дифференцируемой, если приращение имеет вид линейная функция
от дх + о малое от дх
\end{defin}

\begin{theor}
Функция дифференцируема в точке эквивалентно существованию меньшей нуля
производной в этой точке
\end{theor}
\textbf{Доказательство} 1. Необходимость (дифф.=>сущ. произв): предел 
определения производной равен А +о(дх)/дх=А\\
2. Достаточность. Пусть существует производная, равная перделу отношения. 
Тогда мы можем ввести альфа(х ноль, дх) стремится к 0$\square$
\begin{defin}
Линейная часть производной в точке - дифференциал
\end{defin}
\textbf{Замечание.} Непрерывность - необходимое, но недостаточное устловие 
дифференцируемости. Пример - модуль в нуле.
\subsection{Основные правила дифференцирования}
Линейность, дифференциал произведения и деления
\begin{theor}
О производной сложной функции
\end{theor}
\textbf{Доказательство} $\square$
Формула дифференциала функции имеет вид

,

где  - дифференциал  независимой переменной.

Пусть теперь дана сложная (дифференцируемая) функция , где , . Тогда по формуле 
производной сложной функции находим

,

так как .

Итак, , т.е. формула дифференциала имеет один и тот же вид для независимой
переменной  и для промежуточного аргумента , представляющего собой 
дифференцируемую функцию от .

Это свойство принято называть свойством инвариантности формулы или формы 
дифференциала. Заметим, что производная этим свойством не обладает.
\subsection{Геометрический смысл и касательные}
\subsection{Производная обратной функции}
\subsection{Логарифмическая производная}
$y'=y\ln' y$
\subsection{Приближенные вычисления через линенйные приближения}
$\Delta x\approx dx\Rightarrow f(x_0)\approx+f'(x_0)(x-x_0)$
\subsection{Производные и дифференциалы высших порядков}
Формула Лейбница: $$\boxed{(uv)^{n}=\sum\limits^n_{k=0}C_n^ku^{(k)}v^{n-k}}$$
\subsection{Теоремы о функциях, дифференцируемых на отрезках}
\begin{theor} (Ферма)\\

\end{theor}
\textbf{gg}    $\square$
\begin{theor} (Ролля)

\end{theor}
\textbf{Доказательство.}  $\square$
\begin{theor} (Лагранжа)
\end{theor}
\textbf{Доказательство.}  $\square$
\begin{theor} (Коши)

\end{theor}
\textbf{Доказательство.}  $\square$


\subsection{Ряды Тейлора}
\section{Неопределенное интегрирование}
\subsection{Табличное интегрирование}
\subsection{Замена переменных}
\subsection{Интегрирование по частям}
\subsection{Интегрирование рациональных функций}
\subsection{Интегрирование иррациональных функций}
\subsection{Интегрирование тригонометрических функций}
\section{Определенный интеграл}
\subsection{Определение по Риману}
\subsection{Суммы Дарбу}
\subsection{Вычисление длин, площадей, объемов}











\section{Методы опр инт}
\textbf{Пример 1} $\int\limits$
\begin{thebibliography}{}
\bibitem{z}
Зорич В. А. Математический анализ
\bibitem{f}
Фихтенгольц Г. М. Курс дифференциального и интегрального исчисления
\bibitem{k}
Куприн А. В. Неопределенный интеграл за 6 практических занятий

\end{thebibliography}

\end{document}
