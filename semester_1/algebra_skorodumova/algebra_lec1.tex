\begin{abstract}
    Данные записки основаны на лекциях, прочитанных Е. А. Скородумовой в
	первом семестре 2021-2022 учебного года. При написании этого конспекта мы 
	стремились передать логическую структуру алгебры, поэтому нам пришлось
	изменить подачу маетриала в сторону большей теоретической строгости. 
	Большое количество задач во второй части соответствует реальному ходу 
	лекций и семинаров. Я благодарю всех, кто указал на ошибки и опечатки и 
	так или иначе сделал эти  записки лучше.  
\end{abstract}


\section{Системы линейных алгебраических уравнений}
Общий вид СЛАУ из m уравнений и n неизвестных:\\
\begin{equation*}
\begin{cases}
a_{11}x_1+a_{12}x_2+\dots+a_{1n}x_n=b_1\\
a_{21}x_1+a_{22}x_2+\dots+a_{2n}x_n=b_2\\
\ldots \\
a_{m1}x_1+a_{m2}x_2+\dots+a_{mn}x_n=b_m\\
\end{cases}
\end{equation*}
$a_{ij}$ - коэффициенты, $x_i$ - переменная, $b_j$ - свободные члены.

В матричном виде:\\
\begin{equation*}
    \begin{pmatrix} 
a_{11} & a_{12} & \ldots & a_{1n}&\multicolumn{1}{|c}{b_1}\\
a_{21} & a_{22} & \ldots & a_{2n}&\multicolumn{1}{|c}{b_2}\\
\vdots & \vdots & {} & \vdots &\multicolumn{1}{|c}{}\\
a_{m1} & a_{m2} & \ldots & a_{mn} &\multicolumn{1}{|c}{b_m}
    \end{pmatrix}
\end{equation*}
Расширенная матрица - со столбцом свободных членов.


\begin{defin}
Элементарные преобразования СЛАУ:\\
1. Перестановка строк\\
2. Прибавление к строке строки, умноженной на число\\
3. Умножение строки на число
\end{defin}
\begin{defin}
СЛАУ с одинаковыыми решениями - эквивалентны
\end{defin}
\begin{theor}
Элементарные преобразования СЛАУ переводят её в эквивалентную ей систему
\end{theor}
\textbf{Доказательство.} Пусть А - матрица системы, Х - столбец корней 
системы, В - столбец свободных членов. Тогда решение системы можно записать
как $AX=B$. Пусть теперь L - матрица элементарного преобразования системы. 
Поскольку при умножении слева она преобразовывает только строки, её действие 
на расширенную матрицу системы имеет вид $L(A|B)=(LA|LB)$. Таким образом, в 
решении системы $LAX=LB$ элементарные преобразования не затрагивают корни
системы, то есть система остается эквивалентной. $\square$

\begin{defin}\makebox[5pt]{}\\
СЛАУ совместна, если она имеет хотя бы одно решение\\
СЛАУ определенная, если имеет ровно одно решение\\
СЛАУ однородная, если $\forall n: b_n=0 $\\
$\forall n: x_n=0 $ - трииальное решение однородной СЛАУ
\end{defin}
\begin{example} Рассмотрим систему 
$\begin{cases}
x+y-z=-4\\
x+2y-3z=0\\
-2x-2z=16
\end{cases}$
\end{example}
Ей соответсвует матрица \\
\begin{equation*}
 \begin{pmatrix}
  1&1&-1&\multicolumn{1}{|c}{-4}\\
  1&2&-3&\multicolumn{1}{|c}{0}\\
  -2&0&-2&\multicolumn{1}{|c}{16}
\end{pmatrix}   
\sim
\begin{pmatrix}
  1&1&-1&\multicolumn{1}{|c}{-4}\\
  0&1&-2&\multicolumn{1}{|c}{4}\\
  0&0&0&\multicolumn{1}{|c}{3}
\end{pmatrix}
\end{equation*}\\
Отсюда получаем 0=3, следовательно система несовместна.
  
\section{Матрицы}
\subsection{Определение матриц}
\begin{defin}
Матрица A размера $m\times n$ - таблица из m строк и n столюцов, состоящая из
чисел или выражений $a_{ij}$ - элементов матрицы.\\ i - номер строки, j - 
номер столбца.   
\end{defin}

\begin{equation*}
A = (a_{ij}) = ||a_{ij}|| = \left(
\begin{array}{cccc}
a_{11} & a_{12} & \ldots & a_{1n}\\
a_{21} & a_{22} & \ldots & a_{2n}\\
\vdots & \vdots & \ddots & \vdots\\
a_{m1} & a_{m2} & \ldots & a_{mn}
\end{array}
\right)
\end{equation*}

\begin{defin}
Если $A=n \times n$ и $\forall i\ne j: a_{ij}=0$, то матрица A - диагональная.\\
Если матрица диагональная и $a_{ii}=1$, то она - единичная.\\
Если $a_{ij}=a_{ji}$ - матрица симметрическая.\\
Если $a_{ij}=-a_{ij}$ - матрица кососимметрическая
\end{defin}
Также, есть матрицы-столбцы и матрицы-строки.
\begin{defin}
Матрица называется ступенчатой, если:\\
1) Номера первых ненулевых элементов (ненулевых) строк образуют возрастающую 
последовательность\\
2) Нулевые строки, если они есть, стоят в конце.  \\
Квадратная матрица называется треугольной, если она ступенчатая, над главной 
диагональю включительно находятся числа, под главной диагональю находятся нули.
\end{defin}

\subsection{Сложение матриц}
\begin{defin}
Суммой матриц $A=(a_{ij})$ и $B=(b_{ij})$ называется матрица\\
$C=A+B=(c_{ij})=(a_{ij}+b_{ij})$ 
\end{defin}
Матрицы одного размера образуют абелеву группу по сложению (4 свойства).

\subsection{Умножение матриц}
\begin{defin}
Умножение матрицы на число: $C=\alpha A=(c_{ij})=(\alpha a_{ij})$
\end{defin}
Умножение матрицы на число обладает свойствами:\\
1. $\alpha(\beta A)=(\alpha \beta)A$ \\
2. $\alpha(A+B)=\alpha A+\alpha B$\\
3. $(\alpha+\beta)A=\alpha A+\beta A$\\
4. $\exists 1: 1A=A$\\
5. $(-1)A=-A$

\begin{defin}
Умножение матрицы на матрицу производится по правилу строка на столбец:
$A_{m\times n}\times B_{n\times k}=C_{n\times n}=(c_{ij})$

\end{defin}
\begin{equation*}
\boxed{c_{ij}=\sum\limits^n_{k=1} a_{ik} b_{kj}}    
\end{equation*}
Свойства умножения матриц:\\
1. $AB\ne BA$ (но иногда коммутируют)\\
2. A(BC)=(AB)C\\
3. (А+В)С=АС+ВС\\
4. С(А+В)=СА+СВ\\
5. $\alpha AB=(\alpha A)B=A(\alpha B)$\\
6. Иногда есть обратная матрица $E: AA^{-1}=A^{-1}A=E$ \\
7. Единичная матрица: $AE=EA=A$\\

$\sum\limits^n_{i=1}\alpha_i A_i $ - линейная комбинация матриц.


\subsection{Транспонирование}
\begin{defin}
Транспонированная матрица $A^\mathrm{T}$ - матрица, у которой столбцы и 
строки переставлены местами.\\
$A^\mathrm{T}=(a_{ij})^\mathrm{T}=(a_{ji})$
\end{defin}
Cвойства транспонирвоания:\\
1. Если матрица A симметрична, то $A^T=A$\\
2. $(A^T)^T=A$\\
3. $(A+B)^T=A^T+B^T$\\
4. $(AB)^T=B^TA^T$\\


\subsection{Матрицы элементарных преобразований}
\begin{theor}
Умножение матрицы А слева (справа) на матрицы элементарных преобразований
эквивалентно применению соответствующих преобразований к строкам (столбцам)
матрицы А.
\end{theor}
$P_{ij}=\left(
\begin{array}{ccccccc}
1&&&&&&\\
&\ddots&&&&&\\
&&0&\ldots&1&&(i)\\
&&\vdots&\ddots&\vdots&&\\
&&1&\ldots&0&&(j)\\
&&&&&\ddots&\\
&&&&&&1
\end{array}
\right)$ - меняет местами i-тую и j-тую строки.\\
$D_i(\alpha)=
\left(\begin{array}{ccccc}
1&&&&\\
&\ddots&&&\\
&&\alpha&&(i)\\
&&&\ddots&\\
&&&&1
\end{array}\right)
$ - умножает i-тую строку на $\alpha$.\\

$L_{ij}(\alpha)=
\left(\begin{array}{ccccccc}
1&&&&(j)&&\\
&\ddots&&&&&\\
&&1&\ldots&\alpha&&(i)\\
&&&\ddots&\vdots&&\\
&&&&1&&\\
&&&&&\ddots&\\
&&&&&&1
\end{array}\right)
$ - прибавляет к i-той строке j-тую строку, умноженную на $\alpha$.\\
\textbf{Доказательство} - прямым умножением на матрицу $\square$.\\
\textbf{Предложение.} \textit{Строки матрицы А’, полученные через элементарные
преобразования матрицы А, линейно выражаются через строки матрицы А}\\
\textbf{Доказательство.} Очевидно из определения элементарных преобразований
$\square$.
\begin{theor} \label{priv}
Всякая матрица может быть приведена к ступенчатому виду путем элементарных
преобразований строк.
\end{theor}
\textbf{Доказательство.} Если данная матрица нулевая, то она уже ступенчатая.
Если она ненулевая, то пусть $j_1$ –номер ее первого
ненулевого столбца. Переставив, если нужно, строки, добьемся того,
чтобы $a_{1j_1}\ne0$. После этого прибавим к каждой строке, начиная со второй,
первую строку, умноженную на подходящее число, с таким
расчетом, чтобы все элементы $j_1$-го столбца, кроме первого, стали
равными нулю. Мы получим матрицу, у которой верхняя строка выступает своим
ненулевым элементом над нулями. Теперь применим этот алгоритм к матрице, 
которая осталась под этой строкой, и так далее. В итоге получим треугольную 
матрицу. $\square$\\
\textbf{Следствие.} Любую СЛАУ можно привести к ступенчатому виду через 
элементарные преобразоания строк соответствующей матрицы.
\subsection{Определитель}
\begin{defin}
Опрееделитель квадратной матрицы А - число, равное
\end{defin}
\begin{equation*}
    det \left(
\begin{array}{cccc}
a_{11} & a_{12} & \ldots & a_{1n}\\
a_{21} & a_{22} & \ldots & a_{2n}\\
\vdots & \vdots & \ddots & \vdots\\
a_{n1} & a_{n2} & \ldots & a_{nn}
\end{array}
\right) = \sum \limits^{n!}(\pm a_{1i_1}a_{2i_2}\ldots a_{ni_n})
\end{equation*}\\
Определитель находится через перестановки индексов. Так как порядок матрицы n,
то всего имеется n! перестановок $\sigma \colon \{1,2 \ldots i \ldots n\} 
\mapsto \{w_1, w_2 \ldots w_i \ldots w_n\}$. В произведении $\prod a_{ij}$ 
первый индекс пробегает числа $\{1,2 \ldots n\}$, а второй - результат 
применения перестановки $\sigma$ к этому множеству.\\ Если в перестановке 
$\sigma$ имеет место $i<j\Rightarrow w_i>w_j$, то в перестановке есть инверсия.
Если в перестановке четное число инверсий, то перед произведением элементов 
матрицы ставится плюс, в противном случае - минус.

Матрица называется вырожденной, если её определитель равен 0. 
\begin{defin}
Дополнительным минором $M_{ij}$ к элементу $a_{ij}$ квадратной матрицы A 
называется определитель, получившийся вычеркиванием i-той строки и j-того 
столбца определителя А.
\end{defin}
\begin{defin}
Алгебраическим дополнением $A_{ij}$ к элементу $a_{ij}$ квадратной матрицы А 
называется произведение $(-1)^{(i+j)}$ на дполнительный минор к этому элементу.
\end{defin}
В таких обозначениях определитель имеет вид
\begin{equation*}
    \boxed{det A = \sum \limits^n_{j=1}(-1)^{i+j}a_{ij}M_{ij}=\sum 
	\limits^n_{j=1}a_{ij}A_{ij}} \eqno (i=const)
\end{equation*}
\textbf{Свойства определителя:}\\
1. Если определитель содержит нулевой столбец или строку, то он равен 0.\\
2. При умножении столбца или строки на число определитель умножится на это
число.\\
3. Если поменять местами какие-либо две строки или столбца определителя, то
определитель изменит знак.\\
4. Если матрица уможена на число, то определитель умножится на это число в 
степени порядка матрицы.\\
5. Если к какой-либо строке прибавить другую строку, умноженную на число, 
определитель не изменится.\\
6. Если у двух квадратных матриц A и B совпадают все строки кроме i-той, то 
для матрицы C, у которой i-тая строка является суммой i-тых строк матриц А и В,
а все остальные строки совпадают со строками А и В, имеет место $|A|+|B|=|C|$ \\
7. $|A|\times|B|=|A\times B|$\\
8. $|A^T|=|A|$\\
9. $|A|=0$ $\Leftrightarrow$ строки или столбцы линейно зависимы\\
10. $|A^{-1}|=|A|^{-1}$
\subsection{Обратная матрица}
\begin{defin}
Матрица $B=A^{-1}$ - обратная к матрице $A$, если
$$AB=BA=E$$
\end{defin}
\begin{theor} \label{edobm}
(о единственности обратной матрицы)\\
Если A - квадратная и для некоторой B имеет место AB=E, то $B=A^{-1}$
\end{theor}
\textbf{Доказательство.} Как видно из свойств умножения матриц, обратимые
квадратные матрицы образуют (некоммутативную) группу. В разделе о группах 
доказано, что в любой группе противоположный элемент единственный. Отсюда
следует доказательство единственности противоположной матрицы. $\square$

\begin{theor} \label{chuzh}
(о чужих алгебраических дополнениях)\\
Для любой кадратной матрицы сумма произведений элементов одной строки на
алгебраические дополнения другой строки равна нулю:
\end{theor}
$$\sum \limits ^n _{j=1} a_{ij}A_{kj}=0 \eqno k \ne i$$
\textbf{Доказательство.} Рассмотрим определитель матрицы В, которая получается
из матрицы А заменой элементов k-ой строки на элементы i-ой строки. Поскольку
это определитель с двумя равными строками, то он равен нулю. Получим его через
разложение по k-той строке: 
$$det(B)=\sum\limits_{k=1}^nb_{kj}B_{kj}=0$$
Заметим, что у матриц А и В одинаковые алгебраические дополнения в
k-той строке: $A_{kj}=B_{kj}$, а k-тая строка матрицы B по условию равна i-той 
строке матрицы A: $b_{kj}=a_{ij}$. Отсюда имеем
$$\sum\limits_{k=1}^nb_{kj}B_{kj}=\sum\limits_{k=1}^na_{ij}A_{kj}=0\eqno
\square $$

\subsubsection{Метод присоединенной матрицы}
\begin{defin}
Присоединенной матрицей к квадратной матрице А называется транспонированная
матрица алгебраических дополнений.
\end{defin}
$$\Tilde{A}=(A_{ij})^T$$
\begin{theor}\label{prism}
Если матрица квадратная и невырожденная, то у неё есть обратная матрица, равная 
\end{theor}
$$A^{-1}=\frac{\Tilde{A}}{det A}$$
\textbf{Доказательство}. Рассмотрим следующее выражение: 
$\frac{A\Tilde{A}}{det A}$. Так как матрица $\Tilde{A}$ транспонирована, то 
каждая строка матрицы А умножается на столбец алгебраических дополнений к
некоторой строке. По теореме (\ref{chuzh}), это произведение будет равно нулю
везде, кроме диагонали. В самой диагонали имеем разложение матрицы по строке, 
поэтому элемент матрицы, стоящий на диагонали, равен её определителю. Отсюда 
делаем вывод, что $A\Tilde{A}/det A=E$. Но по теореме (\ref{edobm}) если AB=E,
то $B=A^{-1}$. Отсюда следует, что $\frac{\Tilde{A}}{det A}$ - обратная 
матрица для А. $\square$\\
\textbf{Следствие} (теорема об обратимости). Квадратная матрица А имеет 
обратную матрицу $\Leftrightarrow det A \ne 0$  


\subsubsection{Метод Гаусса для поиска обратной матрицы}
Рассмотрим квадратную матрицу А с присоединенной к ней единичной матрицей Е.
Будем преобразовывать её таким образом, чтобы из матрицы А получилась единичная
матрица; тогда присоединенная матрица перейдет в обратную матрицу $A^{-1}$.
\begin{example}
Рассмотрим матрицу
\end{example}
$A=\left( \begin{array}{cc}
    1 &2\\
     3&4 
\end{array} \right)$ . Присоединим к ней Е и проведем элементарные 
преобразования:
$$\Gamma =\left( \begin{array}{ccccc}
    1 &2&\multicolumn{1}{|c}{1}&0 \\
     3&4&\multicolumn{1}{|c}{0}&1 
\end{array} \right)_{II-3I} \sim 
\left( \begin{array}{ccccc}
    1 &2&\multicolumn{1}{|c}{1}&0 \\
     0&-2&\multicolumn{1}{|c}{-3}&1 
\end{array} \right)_{-2II} \sim
\left( \begin{array}{ccccc}
    1 &2&\multicolumn{1}{|c}{1}&0 \\
     0&1&\multicolumn{1}{|c}{3/2}&1/2 
\end{array} \right)_{I-2II} \sim
\left( \begin{array}{ccccc}
    1 &0&\multicolumn{1}{|c}{-2}&1 \\
     0&1&\multicolumn{1}{|c}{\frac{3}{2}}&-\frac{1}{2} 
\end{array} \right)
$$
Таким образом, $A^{-1}=\left( \begin{array}{cc}
-2&1\\
3/2&-1/2
\end{array} \right)$
\begin{theor}
(о методе Гаусса-Жордана)\\
Любая квадратная невырожденная матрица может быть преобразована в единичную 
элементарными операциями над строками или столбцами. 
\end{theor}
\textbf{Доказательство.} По теореме (\ref{priv}), любую матрицу можно привести 
к ступенчатому виду. Так как матрица невырождена, у неё нет нулевых строк или 
столбцов. Следовательно, в первом столбце есть элемент, и он находится в самом
верху, т.к. матрица ступенчатая. Избавимся от лежащих ниже элементов через 
прибавленеи первой строки, умноженную на подходящее число, а затем умножим
первую строку так, чтобы в первом столбце получилась единица. Повторим эту 
операцию со всеми столбцами последовательно. Предыдущие столбцы не изменятся, 
так как в тех строках, которые мы используем, в этих столбцах стоят нули. 
В итоге мы получили единичную матрицу. $\square$


\subsection{Ранг матрицы}
\begin{defin}
Минором k-того порядка произвольной матрицы А называется определитель, 
составленный из элементов матрицы, расположенных на пересечении каких-либо 
k строк и k столбцов.
\end{defin}
\begin{defin}
Ранг матрицы А - наибольший порядок ненулевого минора.
\end{defin}
\begin{defin}
Базисный минор матрицы А - любой из максимально возможных ненулевых миноров.
\end{defin}
\begin{defin}
Матрицы А и В эквивалентны, если $A \mapsto B$ - композиция элементарных 
преобразоаний. 
\end{defin}

\begin{theor}\label{rgconst}
Ранг матрицы не изменяется при элементарных преобразованиях строк матрицы
\end{theor}
\textbf{Доказательство.} Из определения элементарных преобразований следует,
что строки матрицы A’, полученной из матрицы A каким-либо элементарным 
преобразованием, линейно выражаются через строки матрицы A. Но так как матрица
A может быть получена из A’ обратным
элементарным преобразованием, то и, наоборот, ее строки линейно выражаются 
через строки матрицы A’. Таким образом, системы
строк матриц A и A’ эквивалентны и, следовательно, ранги этих
матриц равны. $\square$
\begin{theor}\label{rang}
Ранг матрицы равен числу ненулевых строк любой ступенчатой матрицы, к которой
она приводится элементарными преобразованиями строк.
\end{theor}
\textbf{Доказательство} По только что доказанной теореме, ранг матрицы не 
меняется при элементарных преобразованиях, поэтому нам достаточно доказать, 
что ранг ступенчатой матрицы равен числу ее ненулевых строк. Для этого,
в свою очередь, достаточно доказать, что ненулевые строки ступенчатой матрицы
линейно независимы.\\
Предположим, что линейная комбинация ненулевых строк ступенчатой матрицы с 
коэффициентами $\lambda_1, \lambda_2...\lambda_r$ равна нулю. Рассматривая
$j_1$-ю координату этой линейной комбинации, находим, что $\lambda_1a_1j_1 = 0$,
откуда $\lambda_1 = 0$. Рассматривая, далее, $j_2$-ю координату с учетом того,
что $\lambda_1 = 0$, находим, что $\lambda_2a_2j_2 = 0$, откуда
$\lambda_2 = 0$. Продолжая так дальше, получаем, что все коэффициенты 
$\lambda_1, \lambda_2...\lambda_r$ равны нулю, что и требовалось доказать.
$\square$

\begin{theor}\label{bazmin}
(о базисном миноре матрицы)\\
Минор является базисным при выполнении двух условий:\\
1) Система строк (столбцов), пересекающихся с этим минором, линейна независима
\\
2) любая строка (столбец) матрицы линейно выражается через строки (столбцы) 
этой системы
\end{theor}
\textbf{Доказательство}. Пусть условия выполняются, но минор $s$, 
удовлетворяющий им, не является ненулевым минором наибольшего размера: пусть 
существует минор $s'>s$. Тогда дополнительные строки и столбцы минора $s'$ по 
условию 2 можно линейно выразить через строки и столбцы минора $s$. Но тогда 
минор $s'$ - нулевой, что проиворечит условию. \\
С другой стороны, пусть наибольшим минором являеся минор $s'<s$. Тогда, по 
условию 1, дополнительные строки и столбцы минора $s$ не могут быть выражены 
через строки и столбцы минора $s'$, что противоречит условию максимальности 
минора. $\square$



\begin{theor}
(критерий равенства нулю определителя)\\
Определитель матрицы равен нулю тогда и только тогда, когда её ранг меньше её 
размера.
\end{theor}
\textbf{Доказательство}. Пусть определитель матрицы равен нулю. Приведем 
матрицу к ступенчатому виду. При элементарных преобразованиях определитель не 
сможет стать нулем или перестать быть им. Тогда в нем есть нулевая строка или 
столбец. Отсюда следует, что ранг матрицы меньше её размера по теореме
(\ref{rang}). Обратно, пусть ранг матрицы равен её размеру. Приведем матрицу к 
ступечатому виду; по теореме (\ref{rgconst}), её ранг не изменится, а по 
теореме (\ref{rang}) в ней найдется нулевая строка. Следовательно, ранг матрицы
равен нулю. $\square$


\begin{defin}
Рангом системы векторов называется размер-
ность ее линейной оболочки. 
\end{defin} 


\begin{theor}
(об эквивалентности определений ранга)\\
Ранг матрицы равен рангу системы её строк (столбцов).
\end{theor}
\textbf{Доказательство.} По теореме (\ref{bazmin}), все строки и столбцы
матрицы линейно выражаются через строки и столбцы базисного минора, то есть 
лежат в линейной оболочке строк или столбцов базисного минора. Её размерность 
равна порядку базисного минора, отсюда и следует утверждение теоремы. $\square$



\subsubsection{Метод Гаусса вычисления ранга матрицы} - приводим матрицу к 
ступенчатому виду и смотрим, есть ли там пустые строки или столбцы. По теореме 
(\ref{rang}), ранг матрицы равен количеству ненулевых строк
\subsubsection{Метод окаймляющих миноров.}
\begin{defin}
Пусть $M_s$ - минор порядка s. Окаймляющим минором $M_{s+1}$ для $M_s$ порядка
s+1 будет минор, содержащий $M_s$.
\end{defin}
Сначала берем минор порядка 2 в верхнем левом углу матрицы. Если он не равен 
нулю, то ранг матрицы как минимум два. Тогда переходим к минору порядка 3. 
\\Если он равен нулю, передвигаем вбок второй столбец минора вдоль матрицы, 
пока первый стоит на месте. Если один из миноров - ненулевой, переходим к 
следующему порядку. В противном случае ранг матрицы - 1.  \\
\textbf{Свойства ранга матрицы}\\
1. $\eta(AB)=\eta(A)  (|B|\ne0), \eta(BA)=\eta(B)  (|A|\ne0)$\\
2. $\eta(AB)\leqslant min(\eta(A),\eta(B))$\\
3. $\eta(A^T)=\eta(A)$

\subsection{Матрицы и СЛАУ}
\begin{theor}
(Кронекера-Капелли)\\
СЛАУ совместна тогда и только тогда, когда ранг матрицы системы равен рангу
расширенной матрицы системы.
\end{theor}
\textbf{Доказательство.} Рассмотрим СЛАУ с n неизвестными, причем А - матрица
системы, (A|B) - расширенная матрица системы. 
Если ранг матрицы системы не равен рангу расширенной матрицы системы, то 
возможен только один случай: ранг расширенной матрицы больше ранга матрицы 
системы (так как прибавление нового столбца не может уменьшить ранг матрицы). 
Пусть матрица А приведена к ступенчатому виду: тогда увеличение ранга в матрице 
(А|В) связано с тем, что к нулевой строке матрицы А присоединяется ненулевой 
элемент В. Значит, система имеет уравнение с нулевыми коэффициентами, не равное
нулю, и следовательно, несовместна.  
Обратно, пусть система совместна. Тогда существуют числа 
$x_1,\dots,x_n\in\mathbb R$ такие, что $b=x_1 a_1+\dots+x_n a_n$. 
Следовательно, столбец $b$ является линейной комбинацией столбцов 
$a_1,\dots,a_n$ матрицы A.
Из того, что ранг матрицы не изменится, если из системы её строк (столбцов) 
вычеркнуть или приписать строку (столбец), которая является линейной 
комбинацией других строк (столбцов) следует, что $\operatorname{rang} A = 
\operatorname{rang} A|B$. $\square$\\
Имеем 3 cлучая:\\
$$
\xymatrix{
&& \text{СЛАУ}\ar[dl] \ar@{->}[dr] &\\
& \eta(A|B)=\eta(A) \ar[dr] \ar[dl] &&  \eta(A|B)\ne\eta(A)\text{ (3)}\\
\eta=n \text{ (1)} &&\eta<n \text{ (2)}& 
}
$$
Cлучай 1. Система совместная определенная.\\
Случай 2. Система совместная неопределенная, причем $\eta$ - количество
главных переменных, $n-\eta$ - количество свободных переменных.\\
Случай 3. Система несовместная

\subsubsection{Метод Гаусса решения СЛАУ}
\begin{example}
Рассмотрим систему и приведем её к треугольному виду
\end{example}
$$\left( \begin{array}{cccccc}
    1&2&2&3&\multicolumn{1}{|c}{1} \\
    6&-3&-3&-1&\multicolumn{1}{|c}{-9}  \\
    -7&1&1&-2&\multicolumn{1}{|c}{8}\\
    -3&9&9&10&\multicolumn{1}{|c}{12}
\end{array} \right) \sim
\left( \begin{array}{cccccc}
    1&2&2&3&\multicolumn{1}{|c}{1} \\
    0&15&5&19&\multicolumn{1}{|c}{15}  \\
    0&0&0&0&\multicolumn{1}{|c}{0}\\
    0&0&0&0&\multicolumn{1}{|c}{0}
\end{array} \right)
$$
Система совместная неопределенная, так как ранг расширенной системы больльше 
ранга системы. Её решение в общем виде 
$$\left( \begin{array}{c}x_1\\x_2\\x_3\\x_4 \end{array}\right) =
\left( \begin{array}{c}-1\\1\\0\\0\end{array}\right)+
C_1 \left( \begin{array}{c}-8\\-11\\15\\0 \end{array}\right)+
C_2\left( \begin{array}{c}-7\\-19\\0\\15 \end{array}\right)
$$
где первый столбец - частное решение, столбцы с буквами С - общее решение, 
все вместе - фундаментальное решение. $C_1$ и $C_2$ имеют смысл переменных
$x_3/15$ и $x_4/15$ соответственно.
\subsubsection{Метод Крамера решения СЛАУ}
\textbf{Пример.} Пусть дана система уравнений $$\begin{matrix}
2x+y=5\\x+3z=16\\5y-z=10\end{matrix}$$
Решим по Гауссу. Ответ: $x=1,\quad y=3,\quad z=5$\\
Матрица системы: $$\begin{pmatrix}
2&1&0\\1&0&3\\0&5&-1
\end{pmatrix}$$ её определитель равен $\Delta=-29$\\
Теперь возьмем определитель этой матрицы, в которой первый столбец - столбец
свободных членов: $$\Delta_1=\begin{vmatrix}5&1&0\\16&0&3\\10&5&-1
\end{vmatrix}=-29$$ 
Аналогично, подставляя столбец свободных членов во второй и третий столбец 
матрицы уравнения, получаем $\Delta_2=-87$ и $\Delta_3=-145$.
Получается, $x=\frac{\Delta_1}{\Delta}$, $y=\frac{\Delta_2}{\Delta}$,
$z=\frac{\Delta_3}{\Delta}$
\begin{theor} (о методе Крамера)\\
Пусть $\Delta$ - определитель невырожденной матрицы системы, а $\Delta_1$,
$\Delta_2$ ... $\Delta_i$ ... $\Delta_n$ - определители, полученные заменой 
i-того столбца матрицы на столбец свободных членов.
Тогда $x_i=\frac{\Delta_i}{\Delta}$ 
\end{theor}
\textbf{Доказательство}. Пусть А - матрица системы, Х - столбец корней, В -
столбец свободных членов. Тогда $AX=B$, $X=A^{-1}B=\frac{\Tilde{A}B}{\Delta}$
по теореме (\ref{prism}). Соответственно, каждый корень $x_i$ есть 
произведение i-то транспонированного столбца алгебраическийх дополений на 
столбец свободных членов, деленное на $\Delta$. А это есть разложение матрицы
по i-тому толбцу, замененному столбцом свободных членов. $\square$
