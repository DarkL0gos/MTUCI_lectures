\section{Линейные пространства}
\subsection{Определение и примеры}
\begin{defin}
Множество L элементов (векторов) любой природы называется линейным (аффинным)
пространством над полем K, если:\\
1. L наделено структурой абелевой группы относительно сложения\\
$+\colon L\times L\to L$\\
2. Определено умножение элементов пространства L на элементы поля K:
\\ $\times \colon L\times K \to L$
\end{defin} 
Линейное пространство удовлетворяет следующим аксиомам:\\
$0^0$. Пространство полно относительно опереаций\\
$1^0$. $\forall \Vec{x}, \Vec{y} \in L: \Vec{x}+\Vec{y}=\Vec{y}+\Vec{x}$\\
$2^0$. $\forall  \Vec{x}, \Vec{y}, \Vec{z} \in L: (\Vec{x}+\Vec{y})+\Vec{y}=
\Vec{x}+(\Vec{y}+\Vec{z})$\\
$3^0$. $\exists$ $\Vec{0}\in L$ $\forall \Vec{x}: \Vec{x}+\Vec{0}=\Vec{x}$\\
$4^0$. $\forall \Vec{x}$ $\exists \Vec{x'}: \Vec{x}+\Vec{x'}=\Vec{0}$\\
$5^0$. $\forall \Vec{x} \in L \quad\forall \lambda_1,\lambda_2 \in K: 
(\lambda_1+\lambda_2)\Vec{x}=\lambda_1\Vec{x}+\lambda_2\Vec{x} $\\
$6^0$. $\forall \Vec{x},\Vec{y}\in L\quad \forall \lambda \in K:
(\Vec{x}+\Vec{y})\lambda=\Vec{x}\lambda+\Vec{y}\lambda $\\
$7^0$. $\forall \lambda_1,\lambda_2 \in K \quad \forall \Vec{x} \in L:
\lambda_1(\lambda_2\Vec{x})=(\lambda_1\lambda_2)\Vec{x}$\\
$8^0$. $\exists 1 \in K \quad\forall \Vec{x} \in L: 1\times\Vec{x}=\Vec{x}$

В частности, вектор-ноль единственен, обратные элементы единственны 
(доказательства по определению от противного; для обратных элементов
- по ассоциативности).

Примеры векторных пространств: квадратные матрицы одного порядка,
арифметические прогрессии (размерность 2), многочлены над $\mathbb{R}$
степени меньше или равной n, последовательности с пределом 0, массивы
из n чисел, положительные действительные числа над $\mathbb{R}$ относительно
умножения (сложение) и возведения в степень (умножение).
\subsection{Системы векторов}
Пусть $\Vec{x_1}\ldots\Vec{x_n}\in L,\quad c_1\ldots c_n \in \mathbb R$
\begin{defin}
$c_1\Vec{x_1}+\ldots+c_n\Vec{x_n}$ - линейная комбинация векторов\\
$c_1=c_2=\ldots=c_n=0$ - тривиальная комбинация\\
$c_1^2+c_2^2+\ldots+c_n^2\ne0$ - нетривиальная комбинация
\end{defin}
\begin{defin}
Совокупность векторов называется линейно зависимой, если существует
равная нулю нетривиальная линейная комбинация этих векторов: в противном случае
- линейно независимы
\end{defin}
В частности, в пространстве размерности 3 любые 4 вектора линейно зависимы. 
\subsection{Базис}
\begin{defin}
В линейном пространстве $L$ базис $\Vec{a_1}\ldots\Vec{a_n}$ - упорядоченная
система векторов, удовлетворяющая условиям:\\
1. Система векторов линейно независима\\
2. Каждый вектор в $L$ линейно выражается через базисные векторы.
\end{defin}
Эквивалентное определение: Базис пространства $L$ - линейно независимая
упорядоченная система векторов максимального порядка. 

\begin{theor}
Для каждого вектора существует единственный набор кординат в данном базисе.
\end{theor}
\textbf{Доказательство.} Пусть существуют два набора координат.
Вычтем вектор из вектора в разных координатах, получим ноль.
Поскольку векторы базиса линейно независимы, то комбинация тривиальна, и все
координаты равны. $\square$

\begin{theor}\label{osnlem}
(основная лемма о линейной зависимости)\\
Если векторы $b_1, b_2, …, b_m$ линейно выражаются через векторы
$a_1,a_2, …, a_n$, причем m > n, то векторы $b_1, b_2, …, b_m$ линейно зависимы.
\end{theor}
\textbf{Доказательство.} Пусть
$$b_1 = \mu_{11}a_1 + \mu_{12}a_2 +…+\mu_{1n}a_n$$
\begin{center}
    \ldots
\end{center}
$$b_m =\mu_{m1}a_1 +\mu_{m2}a_2 +…+\mu_{mn}a_n$$
Для любых $\lambda_1, \lambda_2, …, \lambda_m \in K$ получаем
$\lambda_1b_1 +\lambda_2b2 +…+\lambda_mb_m =(\lambda_1\mu_{11}
+\lambda_2\mu_{21} +…+\lambda_m\mu_{m1})a_1
+(\lambda_1\mu_{12} +\lambda_2\mu_{22} +…+\lambda_m\mu_{m2})a_2 +
\ldots +(\lambda_1\mu_{1n} +\lambda_2\mu_{2n} +…+\lambda_m\mu_{mn})a_n$
Рассмотрим систему n однородных линейных уравнений с m неизвестными
$$\begin{cases}
\mu_{11}x_1 + \mu_{21}x_2 +…+ \mu_{m1}x_m =0\\
\mu_{12}x_1 + \mu_{22}x_2 +…+ \mu_{m2}x_m =0\\
\ldots\\
\mu_{1n}x_1 +\mu_{2n}x_2 +…+\mu_{mn}x_m =0 \end{cases}$$
Если $(\lambda_1, \lambda_2, …, \lambda_m)$ – произвольное решение этой
системы, то $\lambda_1b_1 +\lambda_2b_2 +…+\lambda_mb_m =0$.
С другой стороны, из метода Гаусса следует, что эта система имеет ненулевое
решение, поскольку количество уравнений больше количества неизвестных.
Следовательно, векторы $b_1, b_2, …, b_m$ линейно зависимы. $\square$


\begin{theor}
Если базис в $L$ состоит из $n$ векторов, то любой другой базис состоит из
n векторов.\\ 
Если $n<\infty$, то $L$ конечномерно,  $n=dim L$ - размерность пространства.  
\end{theor}
\textbf{Доказательство.} Если бы в пространстве существовали два
базиса из разного числа векторов, то, согласно теореме (\ref{osnlem}),
тот из них, в котором больше векторов, был бы линейно зависим, что противоречит
определению базиса. $\square$


\section{Векторы}
\subsection{Определения и мотивировки}
\begin{defin}
Закрепленные векторы в $\mathbb{R}^2$ и $\mathbb{R}^3$ - направленные отрезки.\\
Закрепленные векторы коллинеарны, если они параллельны.\\
Закрепленные векторы компланарны, если они параллельны некоторой плоскости.\\
Закрепленные векторы равны, если они коллинеарны. сонаправлены и совпадают
по длине.\\
(Свободный) Вектор - класс эквивалентности закрепленных векторов по равенству.
\end{defin}
В частности, $(O,\Vec{e_1},\Vec{e_2},\Vec{e_3})$ - аффинная (общедекартова)
система координат, её оси - абсцисса, ордината, аппликата. 
\subsection{Скалярное произведение векторов}
\begin{defin}
$(\Vec{a},\Vec{b})=\Vec{a}\Vec{b}:=|\Vec{a}|\times|\Vec{b}|\times
\cos{(\widehat{\Vec{a},\Vec{b}})}$  
\end{defin}
Скалярное произведение в ортонормированном координатном (декартовом) базисе - 
произведение вектора-строки на вектор-столбец по правилу матричного умножения.

Если $\Vec{a}=(x_1e_1+y_1e_2+z_1e_3), \Vec{b}=(x_2e_1+y_2e_2+z_2e_3)$, то
$(a,b)=(x_1x_2(e_1,e_1)+y_1y_2(e_2,e_2)+z_1z_2(e_3,e_3))$\\
\textbf{Свойства скалярного произведения векторов:}\\
1. $(a,b)=(b,a)$\\
2. $(\lambda a,b)=\lambda(a,b) $\\
3. $(a+b,c)=(a,c)+(b,c)$\\
Свойства 1-3 выражают линейность\\
4. $(a,a)=|a|^2$
\begin{defin}
$\Vec{a}\perp\Vec{b}\Leftrightarrow(\Vec{a}\ne\Vec{0})\And(\Vec{b}\ne\Vec{0})
\And((\Vec{a},\Vec{b})=0)$ \\
Векторы ортогональны, если они не равны нулю, а их скалярное произведение
равно нулю.
\end{defin}
\begin{defin}
$e_1,e_2,e_3\ldots$ - ортонормированный базис, если выполняются условия:\\
1. $\forall i\ne j:(e_i,e_j)=0$ - ортогональность\\
2. $\forall i: (e_i,e_i)=1$ - нормированность
\end{defin}
Декартовы координаты - координаты в любом ортонормированном пространстве. 


\subsection{Векторное произведение векторов}
\begin{defin}
Вектороное произведение векторов $[\Vec{a},\Vec{b}]=[\Vec{a}\times \Vec{b}]=
\Vec{a}\times \Vec{b}=\Vec{c}:$\\
1. $\Vec{c}\perp\Vec{a} \And \Vec{c}\perp\Vec{b}$\\
2. $\Vec{a},\Vec{b},\Vec{c}$ - правая тройка\\
3. $|\Vec{c}|=|\Vec{a}|\times|\Vec{b}|\times\sin{(\widehat{\Vec{a},\Vec{b}})}$
- модуль векторного произведения имеет смысл площади параллелограмма,
натянутого на два вектора. 
\end{defin}
\textbf{Свойства векторного произведения векторов}\\
1. $[a,b]=-[b,a]$ - антикоммутативность\\
2. $[\alpha a,b]=\alpha[a,b]$\\
3. $[(a_1+a_2),b]=[a_1,b]+[a_2,b]$\\
4. $[a,a]=0$\\
5. Пусть в базисе $(e_1,e_2,e_3)$ векторы имеют координаты
$X=(x_1,y_1,z_1)$ и $Y=(x_2,y_2,z_2)$. Тогда:
$$[X,Y]= \begin{vmatrix}e_1&e_2&e_3\\x_1&y_1&z_1\\
x_2&y_2&z_2
\end{vmatrix}
$$

\begin{defin}
$\Vec{a}||\Vec{b}\Leftrightarrow(\Vec{a}\ne0)\And(\Vec{b}\ne0)\And([\Vec{a},\Vec{b}]=0)$\\
Векторы параллельны, если они не равны нулю, а их векторное произведение равно нулю.
\end{defin}
\subsection{Смешанное произведение векторов}
\begin{defin}
Смешанное произведение векторов а, б, с - скалярное произведение
векторного произведенния а и б, и с\\
$(\Vec{a},\Vec{b},\Vec{c})=([\Vec{a},\Vec{b}],\Vec{c})$
\end{defin}
\textbf{Свойства смешанного произведения:}\\
1. $([a,b],c)=(a,[b,c])$\\
2. $(a,b,c)=(c,a,b)=(b,c,a)$\\
3. $(b,a,c)=-(a,b,c) $\\
4. $(\lambda a,b,c)=\lambda(a,b,c)$\\
5. $(a_1+a_2,b,c)=(a_1,b,c)+(a_2,b,c)$\\
6. Если векторы не равны нулю, а их смешанное произведение равно нулю,
то векторы компланарны.\\
7. Модуль смешанного произведения - объем параллелипипеда,
построенного на векторах\\
8. Смешанное произведение в координатах:
 $(a,b,c)=(a,[b,c])=\begin{vmatrix}x_1&y_1&z_1\\
x_2&y_2&z_2\\x_3&y_3&z_3\end{vmatrix}$

\section{Аналитическая геометрия}
\subsection{Полярная система координат}
Две координаты: угол (по модулю $2\pi$) и длина радиус-вектора
$$\Vec{A}=(x,y)=x\Vec{i}+y\Vec{j}=(r,\varphi)$$
Перевод из полярных в декартовы координаты: $x=r\cos\varphi$,
 $y=r\sin\varphi$\\
Перевод из декартовых координат в полярные: $r=\sqrt{x^2+y^2}$ , 
$\tan\varphi=\frac{y}{x}$\\
$r\in[0,\infty)$ ; $\varphi\in[0,2\pi)$ или $\varphi\in(-\pi,\pi]$

\subsection{Прямая на плоскости}
Прямая плоскости может быть задана следующим образом:
\begin{enumerate}
    \item  Через точку $M=(x_0,y_0)$ и угол наклона $k=\tan\varphi$:
    $$y=kx-kx_0+y_0$$
    \item Если $M=(0,y_0)$, то $y=kx+x_0$
\item Через точку $M=(x_0,y_0)$ и параллельный вектор $\Vec{a}(n,m)$:
 $$\frac{x-x_0}{n}=\frac{y-y_0}{m}$$
\item Через параметрическое уравнение: 
$$\begin{cases}x=nt+x_0\\y=mt+y_0 \end{cases}$$
\item По двум точкам $M(x_1,y_1)$, $M(x_2,y_2)$: сводим задачу к п.3, т.к.
		есть вектор, соединяющий эти точки:
$$\frac{x-x_1}{x_1-x_2}=\frac{y-y_1}{y_1-y_2}$$
\item По точкам пересечения с осями координат $b(0,b)$ и $a(a,0)$: 
$$\frac{x}{a}+\frac{y}{b}=1$$
\item По точке $M=(x_0,y_0)$ и перпендикулярному вектору $\Vec{a}(n,m)$:
$$n(x-x_0)+m(y-y_0)=0$$
\item Общее уравнение прямой на плоскости:
$$Ax+By+C=0$$
\item По углам между осями и расстоянию 
$$x\cos\alpha+y\cos\beta-\rho=0$$
где $\rho(M_0,l)=\frac{|Ax_0+By_0+C|}{\sqrt{A^2+B^2}}$
 
\end{enumerate}

\subsection{Плоскость в пространстве}
Всюду ниже $\alpha$ - плоскость, кроме уравнения пункта 3 
\begin{enumerate}
\item По перпендикулярному вектору и точке:
$$A(x-x_0)+B(y-y_0)+C(z-z_0)=0$$
$M_0(x_0,y_0,z_0)\in\alpha$, $\Vec{n}(A,B,C)\perp\alpha$
\item
$$Ax+By+Cz+D=0$$
 $\Vec{n}(A,B,C)\perp\alpha$\\
$D=-Ax_0-By_0-Cz_0$ - Из предыдущего пункта
\item
$$x\cos\alpha+y\cos\beta+z\cos\gamma-\rho=0$$
$\rho(M_0,\alpha)=\frac{|Ax_0+By_0+Cz_0+D|}{\sqrt{A^2+B^2+C^2}}$
\item По трем точкам: 
$M_1(x_1,y_1,z_1)$, $M_2(x_2,y_2,z_2)$, $M_3(x_3,y_3,z_3)$:\\
Образуем векторы $\Vec{M_1M_2}=(x_2-x_1,y_2-y_1,z_2-z_1)$
и $\Vec{M_1M_3}=(x_3-x_1,y_3-y_1,z_3-z_1)$
Итак, у нас есть точка $M_1$ и два вектора, выходящие из неё. Вспомним, что
$M_0(x_0,y_0,z_0)\in\alpha$, если $([\Vec{M_1M_3},\Vec{M_1M_2}],M_0)=0$. 
Отсюда уравнение плоскости по трем точкам имеет вид определителя:
$$\begin{vmatrix}x-x_1&y-y_1&z-z_1\\x_2-x_1&y_2-y_1&z_2-z_1\\
x_3-x_1&y_3-y_1&z_3-z_1
\end{vmatrix}$$
\item Пусть $a,b,c$ - координаты точек, лежащих на осях $OX,OY,OZ$
соответственно.  Тогда три точки имеют вид
$M_1(a,0,0)$, $M_2(0,b,0)$, $M_3(0,0,c)$, а уравнение имеет вид
$$\begin{vmatrix}x-a&y&z\\-a&b&0\\-a&0&c\end{vmatrix}$$
или же
$$\frac{x}{a}+\frac{y}{b}+\frac{z}{c}=1$$
\item По точке $M_0(x_0,y_0,z_0)\in\alpha$ и двум (непараллельным) векторам
$\Vec{a_1}(n_1,m_1,k_1)$ и $\Vec{a_2}(n_2,m_2,k_2)$; как известно,
два вектора задают плоскость с точностью до параллельного переноса,
поэтому возьмем векторное произведение этих векторв, получим вектор,
перпендикулярный плоскости, и сведем задачу к пункту 1; имеем уравнение
$$\begin{vmatrix}x-x_0&y-y_0&z-z_0\\n_1&m_1,k_1\\n_2&m_2&k_2
\end{vmatrix}$$
\end{enumerate}

\subsection{Прямая в пространстве}
\begin{enumerate}
\item Каноническое уравнение прямой:
$$\frac{x-x_0}{n}=\frac{y-y_0}{m}=\frac{z-z_0}{k}(=t)$$
\item Параметрическое уравнение: 
$$\begin{cases}x=nt+x_0\\y=mt+y_0\\z=kt+x_0 \end{cases}$$
\item По двум точкам:
$$\frac{x-x_1}{x_2-x_1}=\frac{y-y_1}{y_2-y_1}=\frac{z-z_1}{z_2-z_1}$$
\item Через пересечение двух плоскостей:
$$\begin{cases}A_1x+B_1y+C_1z+D_1=0\\A_2x+B_2y+C_2z+D_2=0  \end{cases}$$
$\Vec{n_1}(A_1,B_1,C_1)\ne\Vec{n_2}(A_2,B_2,C_2)$


\end{enumerate}

\subsection{Полезные формулы}
\subsubsection{Каноническое уравнение прямой в (многомерном) пространстве}
\begin{equation*}
    \boxed{\frac{x-x_0}{n}=\frac{y-y_0}{m}=\frac{z-z_0}{k}=\ldots=t}
\end{equation*}
где $M(x_0,y_0,z_0\ldots)$ - точка на прямой, $\Vec{h}(n,m,k\ldots)$ -
вектор, параллельный прямой. 
\subsubsection{Плоскость (прямая), перпендикулярная вектору}
\begin{equation*}
    \boxed{A(x-x_0)+B(y-y_0)+C(z-z_0)=0}
\end{equation*}
где $\Vec{n}(A,B,C)$ - вектор нормали к плоскости,  $M(x_0,y_0,z_0)$ -
точка, принадлежащая плоскости. 
\subsubsection{Расстояние от точки до прямой (плоскости)}
$$\boxed{\rho(M,l)=\frac{Ax_0+By_0+(Cz_0)+D}{\sqrt{A^2+B^2(+C^2)}}}$$
$M(x_0,y_0,z_0)$ -точка, $Ax+By+(Cz)+D=0$ - уравнение прямой (плоскости).
\subsubsection{Угол между векторами}
$$\boxed{\cos{(\widehat{\Vec{a},\Vec{b}})}=\frac{(\Vec{a},
\Vec{b})}{|\Vec{a}|\times|\Vec{b}|}}$$



\section{Кривые второго порядка}
\subsection{Мотивировка и основные определения}
Кривая второго порядка - геометрическое место точек плокости, являющихся
решением уравнения типа $$Ax^2+By^2+2Cxy+2Dx+2Ey+F=0$$
Всякая кривая второго порядка является коническим сечением
\begin{defin}
Эксцентриситет - отношение расстояния от точки кривой до фокуса к расстоянию
от точки кривой до директрисы
\end{defin}


\subsection{Парабола}
Множество точек $$P=\left\{ M(x,y)\mid \frac{\rho(M,F)}{\rho(M,d)}=1 \right\}$$
называется \textbf{параболой}. F - фокус параболы, d - директриса параболы.\\
Каноническое уравнение параболы: $$\boxed{y^2=2px}$$\\
$p=\rho(F,d)$ - параметр параболы. Через него выражется уравнение директрисы:
$d\colon x=-p/2$\\
Эксцентриситет параболы: $\varepsilon=\frac{\rho(M,F)}{\rho(M,d)}=1$

\subsection{Гипербола}
Множество точек 
$$H=\left\{ M(x,y)\mid \frac{\rho(M,F)}{\rho(M,d)}=\varepsilon>1 
\right\} $$ называется \textbf{гиперболой}.\\
Каноническое уравнение гиперболы:
$$\boxed{\frac{x^2}{a^2}-\frac{y^2}{b^2}=1}$$
$y=\pm\frac{b}{a}x$ - асимптоты гиперболы\\
a - действительная полуось, b - мнимая полуось\\
Пусть $F(\pm C,0)$ - фокусы гиперболы. $C=\sqrt{a^2+b^2}$. 
Тогда директрисы гиперболы: $$d_{12}\colon x=\pm\frac{a}{\varepsilon}=
\frac{a^2}{\pm C}$$ $\varepsilon=C/a$



\subsection{Эллипс}
Множество точек 
$$E=\left\{ M(x,y)\mid \frac{\rho(M,F)}{\rho(M,d)}=\varepsilon<1 \right\}
$$ называется \textbf{эллипсом}.\\
Также эллипс:
$$E=\left\{ M(x,y)\mid \rho(M,F_1)+\rho(M,F_2)=const \right\} $$
Каноническое уравнение эллипса:
$$\boxed{\frac{x^2}{a^2}+\frac{y^2}{b^2}=1}$$
a - большая полуось, b - малая полуось\\
$C=\sqrt{a^2-b^2}$, $F(\pm C,0)$\\
$\varepsilon=C/a$, $d:x=a^2/C$\\
При $a=b$, $\varepsilon=0$ - окружность

\subsection{Мнимый эллипс}
$$\frac{x^2}{a^2}+\frac{y^2}{b^2}=-1$$
\subsection{Точка (0,0)}
$$\frac{x^2}{a^2}+\frac{y^2}{b^2}=0$$
\subsection{Две пересекающиеся прямые в точке (0,0)}
$$\frac{x^2}{a^2}-\frac{y^2}{b^2}=0$$
\subsection{Две параллельные прямые}
$$y^2=a^2$$
$$x^2=a^2$$
\subsection{Прямая}
$$y^2=0$$
$$x^2=0$$
\subsection{Пара мнимых прямых}
$$y^2=-a^2$$
$$x^2=-a^2$$



\section{Поверхности второго порядка}
\begin{defin}
Поверхность второго порядка - геометрическое место точек в пространстве
$\mathbb R^3$, задаваемое уравнением второго порядка от x,y,z:
$$Ax^2+By^2+Cz^2+2Dxy+2Exz+2Fyz+Gx+Hy+Iz+J=0$$
\end{defin}
\subsection{Эллипсоид}
$$\frac{x^2}{a^2}+\frac{y^2}{b^2}+\frac{z^2}{c^2}=1$$
При z=0, y=0 или x=0 получаем эллипсы.\\
При a=b, b=c или a=c - эллипсоид вращения.\\
При a=b=c - сфера
\subsection{Однополостный гиперболоид}
$$\frac{x^2}{a^2}+\frac{y^2}{b^2}-\frac{z^2}{c^2}=1$$
При х=0, у=0 - гипербола\\
При z=0 - эллипс\\
Уравнение можно переписать в виде 
$$\frac{x^2}{a^2}-\frac{z^2}{c^2}=1-\frac{y^2}{b^2}$$
$$\left(\frac{x}{a}-\frac{z}{c}\right)\left(\frac{x}{a}+\frac{z}{c}\right)=
\left(1-\frac{y}{b}\right)\left(1+\frac{y}{b}\right)$$
Откуда получаем уравнения плоскостей, пересечение которых образует прямую,
из которых составлен гиперболоид:
$$ \begin{cases}
\mu(\frac{x}{a}-\frac{z}{c})=\lambda(1-\frac{y}{b})\\
\lambda(\frac{x}{a}+\frac{z}{c})=\mu(1+\frac{y}{b})
\end{cases}$$
$\mu^2+\lambda^2\ne0$
\subsection{Двуполостный гиперболоид}
$$-\frac{x^2}{a^2}-\frac{y^2}{b^2}+\frac{z^2}{c^2}=1$$
При x=0 и y=0 - гипербола\\
При $|z|<c$ -$\varnothing$\\
При $|z|>c$ - эллипс
При $a=b$ - гиперболоид вращения 
(сечение плоскость, параллельной OXY - окружность)
\subsection{Эллиптический параболоид}
$$\frac{x^2}{a^2}+\frac{y^2}{b^2}=2z$$
При х=0 или у=0 - парабола\\
При z=0 - точка\\
При z>0 в сечении - эллипс\\
При $a=b$ - эллиптический параболоид вращения
\subsection{Гиперболический параболоид}
$$-\frac{x^2}{a^2}+\frac{y^2}{b^2}=2z$$
При х=0 или у=0 - парабола\\
При $z=const$ - гипербола\\
(кстати, параболического гиперболоида нет)\\
Аналогично однополостному гиперболоиду, получаем уравнения прямых, 
образующих эллиптический параболоид:
$$\left(\frac{y}{b}-\frac{x}{a}\right)\left(\frac{y}{b}+\frac{x}{a}\right)=
2z\times1$$
Откуда получаем уравнения плоскостей, пересечение которых образует 
прямую, из которых составлен гиперболоид:
$$ \begin{cases}
\lambda(\frac{y}{b}-\frac{x}{a})=2\mu z\\
\mu(\frac{y}{b}+\frac{x}{a})=\lambda
\end{cases}$$
\subsection{Конус}
$$\frac{x^2}{a^2}+\frac{y^2}{b^2}-\frac{z^2}{c^2}=0$$
Конус, который продолжается в обе стороны\\
При х=0 и у=0 - пара пересекающихся прямых\\
При $z=const$ - эллипс\\
При $a=b$ - конус вращения\\
При y=0 - 
\subsection{Пара параллельных плоскостей}
$$z^2=c$$
$c>0\colon z=\pm\sqrt{c}$\\
$c=0\colon z^2=0$ - плоскость OXY\\
При с<0 - пара мнимых плоскостей
\subsection{Пара пересекающихся плоскостей}
$$z^2=y^2a^2$$
\subsection{Пара мнимых пересекающихся плоскостей}
$$z^2=-a^2y^2$$
 а также: $x^2=-c^2y^2$ , $\frac{x^2}{a^2}+\frac{y^2}{b^2}=0$
\subsection{Точка (0,0,0)}
$$\frac{x^2}{a^2}+\frac{y^2}{b^2}+\frac{z^2}{c^2}=0$$
\subsection{Мнимый эллипсоид}
$$\frac{x^2}{a^2}+\frac{y^2}{b^2}+\frac{z^2}{c^2}=-1$$
\subsection{Эллиптический цилиндр}
$$\frac{x^2}{a^2}+\frac{y^2}{b^2}=1$$
Цилиндр не ограничен по оси z\\
При $a=b$ - круговой цилиндр
\subsection{Параболический цилиндр}
$y=x^2$
- неограниченная поверхность параболы
\subsection{Гиперболический цилиндр}
$$-\frac{x^2}{a^2}+\frac{y^2}{b^2}=1$$
- неограниченная поверхность гиперболы

\section{Комплексные числа}
\begin{defin}
Комплескное число $z\in\mathbb C$ - упорядоченная пара действительных чисел,
для которой имеет место свойства и две операции:\\
1. $(a_1,b_1)=(a_2,b_2)\Leftrightarrow(a_1=a_2)\And(b_1=b_2)$\\
2. $(x_1,y_1)+(x_2,y_2)=(x_1+x_2,y_1+y_2)$\\
3. $(x_1,y_1)(x_2,y_2)=(x_1x_2-y_1y_2,x_1y_2+x_2y_1)$\\
4. $(x,0)\in\mathbb{R}$
\end{defin}
В частности, $(0,1)^2=(0,-1)^2=(-1,0)$\\
Покажем, что $\mathbb{R}$ подполе в $\mathbb{C}$:\\
$(a,0)=a$, $(b,0)=b)$\\
1. $a=b\Leftrightarrow a=b, 0=0$\\
2. $a+b=(a+b,0)$\\
3. $ab=(ab-0,0+0)$\\
$\Vec{z}=x\Vec{1}+y\Vec{i}$ - векторная форма записи комплексного числа.\\
$$\boxed{z=x+iy}$$ - алгебраическая форма записи

$x=Re(z)$ вещественная часть, $y=Im(z)$ - мнимая часть, 

Обратное число $z^{-1}\colon zz^{-1}=1$ можно найти как
$z^{-1}=\frac{x}{|z|^2}-\frac{iy}{|z|^2}$

\begin{defin}
$\overline{z}=x-iy$ - комплексно сопряженное число для $z=x+iy$
\end{defin}
\textbf{Свойства коплексного сопряжения:}\\
1. $\overline{\overline{z}}=z$\\
2. $\overline{z}=z\Leftrightarrow Im(z)=0$\\
3. $z+\overline{z}=2 Re(z)$\\
4. $z\overline{z}=Re^2(z)+Im^2(z)=|z|^2$\\
5. $\overline{z_1+z_2}=\overline{z_1}+\overline{z_2}$\\
6. $\overline{z_1z_2}=\overline{z_1}\times\overline{z_2}$\\
7. $\overline{z_1/z_2}=\overline{z_1}/\overline{z_2}$, $\overline{z_2}\ne0$

Представим коплексное число как вектор на плоскости с базисными векторами 1
и $i$. Тогда число можно характеризовать двумя величинами - длиной (модулем)
вектора и углом поворота от оси действительных чисел (аргументом). 

Пусть $y=Im(z)$, $x=Re(z)$. Тогда:\\
$r=|z|=\sqrt{x^2+y^2}$ - модуль числа. \\
$tg \varphi = \frac{y}{x}$\\
$\varphi=Arg(z)\in(-\pi,\pi]$ - главное значение аргумента z.\\
$Arg(z)=\{Arg(z)+2\pi n, n\in\mathbb Z\}$\\
$$\boxed{z=r(\cos\varphi+i\sin\varphi)}$$ -
тригонометрическая форма записи комплексного числа.

Из формулы Эйлера $e^{i\varphi}=\cos\varphi+i\sin\varphi$
получаем показательную запись числа: $$\boxed{z=re^{i\varphi}}$$
Имеет место, в частности,
$e^{i0}=1$, $z_1z_2=r_1r_2e^{i(\varphi_1+\varphi_2)}$, $z^n=r^ne^{i\varphi n}$

Фрмула Муавра: $(\cos\varphi+i\sin\varphi)^n=\cos{n\varphi}+i\sin{n\varphi}$

Рассмотрим множество комплексных корней из единицы, то есть таких z,
что $z^n=r^ne^{in\varphi}=1$. 
Очевидно, что для всех таких чисел r=1\\
Пусть n=6. Тогда множество это имеет вид
$$\sqrt[6]{z}=\{e^{i\frac {2\pi k}{6}}, k\in\{0...5\}\}=\{e^0, e^{i\pi1/3},
e^{i\pi2/3}, e^{i\pi3/3}, e^{i\pi4/3}, e^{i\pi5/3} \}$$
Убедиться в том, что это корни из единицы, можно, возведя эти числа в степень
6 и получить $e^{2i\pi k}=e^0=1$
Геометрически, эти числа образуют вписанный в единичную окружность правильный 
шестиугольник, которой одной из вершин находится в действительно единице.

Итак, множество корней степени n из единицы:
$$\varepsilon_n=\left\{e^{(2i\pi k)/n}\mid k\in\{0...n-1\}\right\}$$
Из произвольного числа z:
$$\alpha_n=\left\{\sqrt[n]{r}e^{i(\varphi+2\pi k)/n}\mid
k\in\{0...n-1\}\right\}$$

Тригонометрические функции через комплексные числа: (из формулы
Эйлера):\\
$$\sin z=\frac{e^{iz}-e^{-iz}}{2i}=\frac{1}{i}Sh(iz)$$
$$\cos z=Ch(iz)$$
\begin{example}
Найдем сумму $S=\sin x + \sin{2x}+...+\sin{nx}$
\end{example}
По формуле для комплексного гиперболического синуса выразим эту сумму:
$$S=\frac{e^{ix}+e^{2ix}+...-(e^{-ix}+e^{-2ix}+...)}{2i}$$
Заменим суммы по формуле для геометрических прогрессий и приведем к общему
знаменателю: 
$$S=\frac{e^{ix}(e^{ixn}-1)-e^{-ix}(e^{-ixn}-1)}{2i(e^{ix}-1)(e^{-ix}-1)}$$
Раскроем скобки сверху и приведем к синусам сверху:
$$S=\frac{\sin{xn}+\sin x-\sin{x(n+1)}}{2-e^{ix}-e^{-ix}}$$
Сверху приведем к умножению по формулам для суммы синусов и разности 
косинусов, а снизу сделаем замену по формуле косинуса как комплексного
гиперболического косинуса и применим формулу косинуса двойного угла; итак,
$$S=\frac{\sin(x(n+1)/2)\sin(xn/2)}{\sin(x/2)}$$
