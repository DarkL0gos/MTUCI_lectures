\section{Группы, кольца, поля (29.11.2021)}
\subsection{Группы}
\begin{defin}
Множество М с операцией * называется группой $<M,*>$, если множество
замкнуто относительно операции и выполняются условия:\end{defin}
1. $\forall a,b \in M: a*b=b*a$ (условие абелевости)\\
2. $\forall a,b,c \in M: a*(b*c)=(a*b)*c$\\
3. $\exists e \in M \forall a \in M: a*e=e*a=a$\\
4. $\forall a \in M \exists a^{-1}\in M: a*a^{-1}=e$\\
\textbf{Примеры не групп}: натуральные числа по сложению\\
\textbf{Примеры (абелевых) групп}: целые, рациональные, действительные,
комплексные числа по сложению, векторные пространства по сложению
векторов \\
Примеры неабелевых групп: невырожденные квадратные матрицы над
действительными числами по умножению: ассоциативно, нейтральный
элемент - единичная матрица, обратный элемент - обратная матрица.
Следовательно, нейтральный и обратные элементы единственны (по
свойству группы)\\ 
\textbf{Терминология}\\
Группы мультипликативные: элементы единичные и обратные\\
Группы аддитивные: элементы нулевые и противоположные\\
\begin{theor}
В любой группе существует единственный обратный и нейтральный элемент.
\end{theor}
\textbf{Доказательство}. От противного; для нейтрального элемента - сложим два;
для обратного элемента - по ассоциативности. $\square$\\
$b+(-a)$ - правосторонняя разность, $(-a)+b$ - левостронняя разность.\\
Рассмотрим аддитивную группу вычетов по модулю n
$\mathbb Z_n=\{0,1\ldots n-1\}$.
Таблица Кэли для n=2: \quad
$\begin{matrix}
+&0&1\\0&0&1\\1&1&0
\end{matrix}
$\quad, n=4: \quad $
\begin{matrix}
+&0&1&2&3\\0&0&1&2&3\\1&1&2&3&0\\2&2&3&0&1\\3&3&0&1&2
\end{matrix}
$\\
Как доказать, что операция ассоциативна? А вот так: обычное сложение
ассоциативно, и так как остаток мы берем от результата, то сложение в 
$\mathbb Z$ ассоциативно (аналогично коммутативность)\\
Наличие нейтрального элемента - ноль\\
Обратный элемент - давайте докажем, что он есть! 
А он есть, так как $a+(n-a)_{mod(n)}=0$. Следовательно, $n-a$ - обратный.\\
Пусть $x\in\mathbb Z_n$. Определим $x^m$ как $x+\ldots+x_{mod (n)}$ m раз.
Тогда $$\{x^m\mid m\in \mathbb N,~x\in\mathbb Z_n,~ x\ne0\}=\mathbb Z_n$$
\begin{defin}
n=ord(g) - порядок элемента g в группе G, если $g^n=e$
\end{defin}
Элемент группы \textbf{образующий}, если его порядок равен порядку группы.\\
Если в группе есть образующий элемент, то группа циклическая - все её элементы
порождаются степенями этого элемента.
\begin{example}
Проверить, что мн-во корней из единицы $\varepsilon_k=e^{2i\pi k/n}$\\ $\{
\varepsilon_k\in\mathbb C\colon \varepsilon_k^i=1\}=\{e^{0i},
e^{i\pi/n},e^{2i\pi/n}\ldots e^{(n-1)i\pi/n}\}$ образуют группу по умножению.
\end{example}
Проверим, что множество замкнуто:
$\varepsilon_k\varepsilon_m=e^{2i\pi (k+m)/n}$\\
Проверим ассоциативность и коммутативность: есть, т.к. комплексные числа
ассоциативны и коммутативны.\\
Нейтральный элемент - единица\\
Обратный элемент: $\varepsilon_k\varepsilon_{n-k}=e^{2i\pi
(k+n-k)/n}=e^{2i\pi}=1$\\
Образующие элементы этой группы - $\pm1$, а также числа, у которых $k$
взаимно просто с порядком корня.\\
\begin{example}
Рассмотрим множество $\mathbb H=\{1,-1,i,-i,j,-j,k,-k\}$.
Рассмотрим операцию умножения, заданную првилами ii=jj=kk=-1, ij=k, jk=i, ki=j, 
ji=-k, kj=-i, ik=-j.
\end{example}  Таблица Кэли для этого множества имеет вид
$$\begin{matrix}\times&1&-1&i&-i&j&-j&k&-k\\
1&1&-1&i&-i&j&-j&k&-k\\
-1&-1&1&-i&i&-j&j&-k&k\\
i&i&-i&-1&1&k&-k&-j&j\\
-i&-i&i&1&-1&-k&-k&j&-j\\
j&j&-j&-k&k&-1&1&i&-i\\
-j&-j&j&k&-k&1&-1&-i&i\\
k&k&-k&j&-j&i&-i&-1&1\\
-k&-k&k&-j&j&-i&i&1&-1\\
\end{matrix}$$
Доказательство того, что это группа - через перебор всех операций. 
\subsection{Кольца}
\begin{defin}
Непустое множество R называется кольцом, если множество
замкнуто относительно двух операций (слоежения и умножения) и выполняются
аксиомы:\end{defin}
Аксиомы сложения:\\
1. $\forall a,b \in R : a+b=b+a$ \\
2. $\forall a,b,c \in R: a+(b+c)=(a+b)+c$\\
3. $\exists 0 \in R \forall a \in R: a+0=0+a=a$\\
4. $\forall a \in R \exists -a\in R: a+(-a)=0$\\
Аксиомы умножения:\\
5. $(ab)c=a(bc)$\\
6. $(a+b)c=ac+bc$, $c(a+b)=ca+cb$\\
Итак, от кольца требуется, чтобы оно было абелевой группой по слоению, 
умножение было ассоциативным, и выполнялась дистрибутивность умножения 
относительно сложения. \\
Если умножение коммутативно, то кольцо коммутативное\\
Если в кольцее есть нейтральный элемент по умножению, то кольцо с единицей.\\
Целые, рациональные, действительные, комплексные числа - коммутативные 
кольца с единицей. \\
\textbf{Пример}. $\{m+\sqrt{n}\mid m,n\in\mathbb Z,~n=const\}$ - кольцо?
Оно полно, сложение - коммутативная группа, ассоциативность и коммутативность
умножения есть, дистрибутивность также есть. Сл-но, коммутативное кольцо с
единицей.\\
\textbf{Пример}. Рассмотрим группу вычетов по модулю n и определим умножение 
как остаток от деления произведения на n. 

$\begin{matrix}
\times&0&1&2&3\\0&0&0&0&0\\1&0&1&2&3\\2&0&2&0&2\\3&0&3&2&1
\end{matrix}$ \quad $\begin{matrix}
+&0&1&2&3\\0&0&1&2&3\\1&1&2&3&0\\2&2&3&0&1\\3&3&0&1&2
\end{matrix}$\\
Получилось коммутативное кольцо с единицей.\\

\textbf{Свойства колец}\\
Аддитивная абелева группа:\\
1.1 Существует ноль\\
1.2 Существует противоположный элемент\\
1.3 Существует единственное решение уравнения $a+x=b$\\
Мультипликативная группа:\\
2. $a(b-c)=ab-ac$ Докажем: $b=(b-c)+c$, $ab=a((b-c)+c)$, $ab=a(b-c)+ac$,
$a(b-c)=ab+(-ac)$, $a(b-c)=ab-bc$\\
3. $\forall a\in R: 0a=a0=0$ Докажем: $0=b-b$, $a(b-b)=ab-ab=0$ (по св. 2)\\
4. $\forall a,b\in R: (-a)b=a(-b)=-ab$ Докажем: $-a=0-a$, 
$(-a)b=(0-a)b=0b-ab=-ab$\\
5. Если есть нейтральный элемент по умножению, то $-1a=a(-1)=-a$ Докажем:
$-1a=(0-1)a=0a-1a=0-a=-a$\\
6. Если в кольце c единицей содержится не менее двух элементов, то $0\ne1$. 
Докажем. Пусть 0=1 и существует $0\ne a\ne1$. Тогда $0=0a=1a=a$, получаем 
противоречие.
\begin{defin}
Делители нуля - такие числа в кольце, что $a\ne0$, $b\ne0$ и $ab=0$
\end{defin}
a - левый делитель нуля, b - правый делитель нуля\\
\subsection{Поля}
\begin{defin}
Коммутативное кольцо R с единицей называется полем, если \\
1. В R нет делителей нуля\\
2. Для каждого ненулевого элемента определен обратный элемент по умножению
\end{defin}
Итак, поле это коммутативная группа по сложению и коммутативная группа по 
умножению, с дистрибутивностью умножения относительно сложения.\\ 
\textbf{Замечание (без доказательства).} В \textit{конечном} коммутативном
кольце всякий элемент либо обратим, либо является делителем нуля, поэтому,
чтобы оно было полем, достаточно требовать отсутствия делителей нуля.\\
\textbf{Свойства поля}\\
1. Все свойства коммутативного кольца\\
2. $\exists ! 1: 1a=a1=a$\\
3. $\exists ! a^{-1}: aa^{-1}=a^{-1}a=1$\\
4. Правила для дробей:\\
4.1 $a/b=c/d \Leftrightarrow ad=bc$. Доказательство: умножим на $bd$\\
4.2 $a/b+c/d=(ad+bc)/bd$. Доказательсвто: умножим, а потом разделим на $bd$\\
4.3 $-a/b=a/-b=-(a/b)$. Доказательство: считая, что деление на $b$ есть
умножение на $b^{-1}$, аналогично этому свойству в кольце.\\
\textbf{Пример}. $\mathbb Z$, $\mathbb Q$, $\mathbb R$, $\mathbb C$ - поля ли?
Целые числа - нет, остальные - да \\
\textbf{Пример}. $M=\{m+n\sqrt{2}\mid m,n\in\mathbb Z\}$ - поле? Если нет, то
какое расширение будет полем? Пусть $m_1$, $m_2$ $\in M$.
$m_1/m_2=m_3=m+n\sqrt{2}$, вообще говоря, не лежит в M, но лежат в 
$M'=\{m+n\sqrt{2}\mid m,n\in\mathbb Q\}$ \\
\textbf{Пример}. Рассмотрим поле $\mathbb Z_5$. В нем 3/4=2, 2/3=4, 2*4=3, 2=3*4\\
\textbf{Пример}. $7^{2021}mod(10)=?=7$\\
\textbf{Пример}. Доказать, что в группе G:
$(ab)^{-1}=b^{-1}a^{-1}$; $(a^{-1})^n=a^{-n}$\\
\textbf{Доказательство.} 1. Домножим $b^{-1}a^{-1}$ на $ab$:
$b^{-1}a^{-1}ab=b^{-1}eb=e$,
следовательно они обратные друг другу.
Но $(ab)^{-1}$ обратный по определению, поэтому
$(ab)^{-1}=b^{-1}a^{-1}$.\\
2. По индукции,
$(a^{-1})^na^n=a^{-1}a^{-1}\ldots a^{-1}aa\ldots a=a^{-1}a^{-1}\ldots 
a^{-1}ea\ldots a=e$, 
следовательно, они обратные. 
\\
\textbf{Пример}. Решить 2x+4=0 в $\mathbb Z_6$. Решение. Прибавим 2, 
получим 2х=2. 2 решения: 1 и 4\\
\textbf{Пример}.Решить 6x+5=0 в $\mathbb Z_6$ Решение. Аналогично получаем 6х=3, 
но у этого нет корней.\\
\textbf{Пример}. Решить $x^2+x+6=0$ в $\mathbb Z_{12}$. Решение - перебором 
разных элементов кольца. х=2,5,6,9\\
\textbf{Пример}. Решить 3х+4=0 в $\mathbb Z_{17}$ (поле!). Решение. 
3х=13, $x=13\times 3^{-1}=13\times 6=10$ - решение единственно, поскольку 
обратный элемент в поле единственен. \\
\textbf{Пример.} $2x^2+5x+4=0$, $\mathbb Z_{11}$. Вычисляем дискриминант и
решаем по обычной схеме: D=25-4*2*4=-7=4, $\sqrt{4}=2$
, имеем корни $x_1=(-5+2)/4=-3/4=8/4=2$, $x_2=(-5-2)/4=-7/4=4/4=1$ \\
\textbf{Пример.} Решим систему в $\mathbb Z_{13}$:
$$\begin{cases} 
3x+4y=1\\5x+3y=7
\end{cases}$$ \textbf{Решение}. Из первого вычтем второе, чтобы игрек 
сократился. Получаем х=7. Подставим во второе, получим у=8.\\
\section{Многочлены}
\subsection{Определения и мотивировки}
\begin{defin}
Многочленом (полиномом) $f(x)$ от переменной x над полем $\mathbb F$
называется формальное выражение вида $f(x)=\sum\limits_{k=0}^{n}a_kx^k$
\end{defin}
$a_0,a_1\ldots a_n \in \mathbb F$ - коэффициенты многочлена,
$n\in \mathbb N\cup \{0\}$\\
$\mathbb F[x]$ - множество всех мночленов над полем $\mathbb F$\\
Степень многочлена: $a_n\ne0$, $a\in \mathbb F$, $deg(f(x))=n$ -
наибольшая степень корня, при котором находится ненулевой
коэффициент:
$f(x)=a_0\Rightarrow deg(f(x))=0$. $f(x)=0$ - многочлен без
степени.\\
$f(x)=\sum\limits^n_{k=0} a_kx^k$, $g(x)=\sum\limits^m_{k=0} b_kx^k$\\
Равенство многочленов:
$f(x)=g(x)\Leftrightarrow deg(f)=deg(g) \And \forall k\colon a_k=b_k$\\
Сложение многочленов: поэлeментно в соответствии со степенью корня:
$f(x)+g(x)=h(x)=\sum(a_k+b_k)x^k$\\
\textbf{Свойства сложения:}\\
1. $deg(f+g)\leqslant max(deg(f),(deg(g)))$\\
2. $f+(g+h)=(f+g)+h$\\
3. $\exists 0\colon 0+f=f+0=f$\\
4. $\forall f\in \mathbb F[x] \quad\exists (-f) \in \mathbb F[x]:
f+(-f)=(-f)+f=0$\\
5. $f+g=g+f$\\
Свойства 2-5 означают, что $\mathbb F[x]$ - аддитивная абелева группа.\\
\textbf{Произведение многочленов}
$f(x)=\sum\limits^n a_kx^k$, $g(x)=\sum\limits^m b_kx^k$\\
$f(x)g(x)=h(x)=\sum\limits^{n+m}c_kx^k$, $c_k=\sum\limits_{i+j=k}a_ib_j=
\sum\limits_{i=0}^{k}a_ib_{k-i}$\\
\textbf{Пример.} $(3x^3+5x^2+12x)(2x^2-5x+1)=(6x^5-5x^4+2x^3-55x^2+12x)$\\
\textbf{Свойства умножения}\\
1. $deg(fg)=deg(f)+deg(g)$\\
2. $(fg)h=f(gh)$ - из ассоциативности поля (сложный логический
вывод...)\\
3. $(f+g)h=fh+gh$ - из дистрибутивности поля\\
4. $\exists 1\colon 1f=f1=f $\\
5. $fg=gf$\\
6. $\forall h(x)\ne0, f(x)\ne0\colon fh\ne0 $ \\
Свойства 2-6 и свойства сложения говорят о том, что $\mathbb F[x]$ -
коммутативное кольцо с единицей без делителей нуля. \\
Поле ли это? Ответ - нет. Пример: $f(x)=x^5$ не имеет обратного, ибо 1/5 не 
является целым числом. Хотя частные случаи возможны, в общем случае деление 
с остатком.\\
\textbf{Разделить} $f(x)$ на $g(x)$ значит найти такие $q(x)$ и $r(x)$, что 
$gq+r=f$ и ($deg(r)<deg(g)$ или $r=0$).\\
\textbf{Пример}. Разделим $2x^2-5x+1$ на $3x^3+5x^2+12x$. Получим $q=0$, 
так как степень делителя больше чем степень делимого, $r=2x^2-5x+1$. 
\#действительно.\\
\textbf{Пример}. Деление, если степень числителя больше степени знаменателя -
в столбик. \\
\textbf{Пример}. Поделить $2x^4+3x^3+4x+1$ на $3x^2+x+1$ в $\mathbb Z_5$. 
Ответ: $q=4x^2+3x+1$, $r=0$\\
\textbf{Задача.} Найти все обратимые многочлены. Во-первых, многочлены 
нулевой степени обратимы (т.к. они лежат в поле). Если степень больше нуля - ?
\begin{theor}
Обратный многочлен по умножению существует тогда и только тогда, когда
степень многочлена равна нулю. 
\end{theor}
\textbf{Доказательство}. Если степень многочлен равна нулю, то он -
константа из поля, следовательно, обартим. Если степень многочлена $f$ 
больше нуля, то предоположим, что сущестует такой $h$, что $fh=1$. Тогда
$deg(fh)=0$, но мы знаем, что $deg(fh)=deg(f)+deg(h)$. Но степень многочлена 
не может быть меньше нуля, и мы приходим к противоречию. $\square$

\begin{defin}
Многочлен h является общим делителем f и g, если f и g делятся на h.
\end{defin}
\textbf{Свойства делимости многочленов}\\
1. $f\vdots g \And g\vdots h \Rightarrow f\vdots h$\\
2. $f\vdots g \And h\vdots g \Rightarrow (f+h)\vdots g$\\
3. $f\vdots g \Rightarrow \forall h \in \mathbb F[x] \colon fh\vdots g$\\
4. $\forall k\in\{1\ldots n\}~f_k\vdots g \Rightarrow \forall k\in\{1\ldots n\}
\forall h_k\colon h_1f_1+h_2f_2+\ldots+f_nh_n\vdots g$
(линейная комбинация многочленов, делящихся на g, делится на g)\\
5. $\forall f\in \mathbb F[x] \forall a\in \mathbb F \colon f\vdots a$\\
6. $f\vdots g \Rightarrow f\vdots cg, c\ne0, c\in\mathbb F$\\
7. $deg(f)=deg(g) \And f\vdots g\Rightarrow g=cf,~c\ne0$\\
8. $f\vdots g \And g\vdots f \Rightarrow g=cf$\\
9. У $f$ и $cf$ все делители совпадают.\\
\textbf{Пример}. Пусть $f(x)=x^3-x-6$, $g(x)=x^2+x-6$ - два многочлена.
Как описать все их делители с точностью до умножения на константу из поля?
\begin{defin}
Наибольший общий делитель НОД(f,g)=$D(f,g)$ многочленов f(x) и g(x) 
называется такой многочлен h(x), который делит f(x) и g(x) и делится на 
любой общий делитель f(x) и g(x).
\end{defin}
НОД(f,g)=1 $\Leftrightarrow $ f и g взаимно простые.\\



\subsection{Теоремы о многочленах}
\textbf{Задача.} Найти НОД произвольных многочленов над произвольным полем. 
Рассмотрим \textbf{Алгоритм Евклида} для многочленов

Пусть f,g - многочлены. Определим среди них больший по степени;
если степени равны, выкинем эти члены и рассмотрим то, что
осталось.\\
Итак, пусть f больше g. Разделим f на g, обозначим остаток как
$r_1$. Поделим g на $r_1$, обозначим остаток как $r_2$. Далее
поделим $r_1$ на $r_2$, получаем остаток $r_3$, и так далее. Таким
образом, мы имеем последовательнсоть остатков $r_i$. Пусть $r_k=0$ -
последний остаток. Тогда $r_{k-1}$ - искомый НОД

\begin{theor} (об алгоритме Евклида)\\
Для любых многочленов a, b существует наибольший общий делитель d, и он может
быть представлен в виде $d=au+bv$, где u, v – какие-то элементы кольца
многочленов. 
\end{theor}
\textbf{Доказательство.} Если $b=0$, то $d=a=1a+0b$. Если a делится на b, то
$d=b=0a+1b$. В противном случае разделим с остатком a на b, затем b на 
полученный остаток, затем первый остаток на второй остаток и т. д. Так как
нормы остатков убывают, то в конце концов деление произойдет без остатка. 
Получим цепочку равенств:
\begin{center}
  $a=q_1b+r_1$\\
$b=q_2r_1+r_2$\\
$r_1=q_3r_2+r_3$\\
\ldots\\
$r_{n-2} = q_nr_{n-1} + r_n$\\
$r_{n-1} = q_{n+1}r_n$.  
\end{center}
Докажем, что последний ненулевой остаток $r_n$ и есть наибольший общий
делитель элементов a и b.
Двигаясь по выписанной цепочке равенств снизу вверх, получаем последовательно
\begin{center}
$r_n | r_{n-1}$\\
$r_n | r_{n-2}$\\
\ldots\\
$r_n | r_1$\\
$r_n | b$\\
$r_n | a$.
\end{center}
где символ $a|b$ означает <<a делит b>>.
Таким образом, $r_n$ - общий делитель элементов a и b. Двигаясь по той же 
цепочке равенств сверху вниз, получаем последовательно
\begin{center}
$r_1 = au_1 + bv_1$\\
$r_2 = au_2 + bv_2$\\
$r_3 = au_3 + bv_3$\\
\ldots\\
$r_n = au_n + bv_n$\\  
\end{center}
где $u_i$, $v_i$ $(i =1, \ldots, n)$ - какие-то элементы кольца 
(например, $u_1=1$, $v_1 = -q_1$). Таким образом, $r_n$ можно представить
в виде $au + bv$.
Отсюда, в свою очередь, следует, что $r_n$ делится на любой общий делитель 
элементов a и b. $\square$\\
\textbf{Замечание.} В равенстве $d=au+bv$,  $deg(u)<deg(b)$, $deg(v)<deg(a)$ \\
\textbf{Следствие}. Многочлены $f,g\in\mathbb F[x]$ взаимно-простые тогда и
только тогда, когда есть такие $u,v\in\mathbb F[x]$, что $fu+gv=1$.
\begin{theor}
Если $f(x)$ взаимно-прост с $\varphi(x)$ и $\psi(x)$, то он взаимно-прост с
$\varphi(x)\psi(x)$  
\end{theor}
\textbf{Доказательство}. Допустим противное: пусть $f=q\varphi\psi$ По условию,
$f$ и $\varphi$ взаимно-просты, поэтому $fu+\varphi v=1$, $q\varphi
\psi u+\varphi v=1$, $\varphi(q\psi u+v)=1$. Значит, $\varphi$ и $(q\psi u+v)$
обратные друг другу. По теореме об обратных многочленах, в таком случае они
являются константами. Но это противоречит тому, что $\varphi$ взаимно-прост с
$f$. Аналогично поступаем с $\psi$. $\square$
\begin{theor}
Если НОД$(f,\varphi)=1$ и $fg\vdots\varphi$, то $g\vdots\varphi$
\end{theor}
\textbf{Доказательство}. Так как $fg\vdots\varphi$, то можно написать равенство
$fg=q\varphi$. По условию, $fu+\varphi v=1$. Домножим на $g$ и заменим согласно
равенству: $q\varphi u+g\varphi v=g$, $\varphi(qu+gv)=g$, следовательно, 
$g\vdots\varphi$. $\square$ 
\begin{theor}
Если $f$ делится на $\psi$ и $\varphi$ и они взаимно простые, то $f$ делится
на $\psi\varphi$.
\end{theor}
\textbf{Доказательство}. $f/\varphi=q_1$. По только что доказанной теореме,
$q_1\vdots \psi$, $q_1=\psi q_2$. Таким образом, $f=q_2\varphi\psi$. $\square$

Рассмотрим деленение многочлена $f(x)$ на $(x-x_0)$, $x_0\in\mathbb F$. Тогда
$f(x)=(x-x_0)q(x)+r(x)$, $deg(q)+1=deg(f)$, $r(x)=const$. Как найти коэффициенты
нового многочлена $q(x)$? Пусть а из ф, б из ку. $a_n=b_{n-1}$,
$a_{n-1}=b_{n-2}-b_{n-1}x_0$, ... $a_2=b_1-b_2x_0$, $a_1=b_0-b_1x_0$,
$a_0=c_0-b_0x_0$.
Выразим из этой штуки b: $b_i=a_i+x_0b_{i+1}$. Схема Горнера. 
\textbf{Пример}. $f=x^4-5x^3+8x^2-5x+3$. Делим на (х-2)
$$\begin{array}{cccccc}
&1&-5&8&-5&3\\2&1&-3&2&-1&1
\end{array}$$ $f(x)=(x-2)(x^3-3x^2+2x-1)+1$\\
Схема Горнера позволяет быстро считать значения многочлена в точке, 
поскольку задействет меньше умножений и сложений. \\
\textbf{Пример}. Поделим в $\mathbb R$ $x^4+3x^3+4x^2-5x-5$ на х+3:
$$\begin{array}{cccccc}
&1&3&4&-5&-51\\-3&1&0&4&-17&0
\end{array}$$ Итак, $f(-3)=0$
\begin{defin}
Если $f(x_0)=0$, то $x_0$ - корень многочлена $f(x)$
\end{defin}
\begin{theor}(Теорема Безу)\\
Остаток от деления многочлена $f(x)$ на $(x-x_0)$ равен $f(x_0)$
\end{theor}
\textbf{Доказательство.} По условию, $f(x)=(x-x_0)g(x)+c_0$, 
тогда $f(x_0)=c_0$. $\square$
\begin{theor}
$x_0\in\mathbb F$ - корень многочлена $f\in\mathbb F[x]$ 
$\Leftrightarrow$ $f(x)\vdots (x-x_0)$
\end{theor}
\textbf{Доказательство.} Пусть $f(x)\vdots (x-x_0)$. Тогда $f(x)=(x-x_0)g(x)$,
и при подстановке $x_0$ дает ноль. Обратно, пусть $f(x_0)=0$. По теореме Безу,
остаток от деления на $(x-x_0)$ равен 0, следователльно, $f(x)\vdots (x-x_0)$.
$\square$ 
\begin{defin}
k - кратность корня $x_0$ многочлена $f(x)$, если $f(x)$ днлится на $(x-x_0)^k$,
но не делится на $(x-x_0)^{k+1}$
\end{defin}
\begin{theor}
(Основная теорема алгебры)\\
У любого многочлена f(x) (степени $\geqslant1$) над полем комплексных чисел 
существует хотя бы один корень.  
\end{theor}
\textbf{Следствие 1}. Любой многочлен степени n над $\mathbb C$ имеет ровно n
корней (с учетом кратности).\\
\textbf{Доказательсво}. Исходный многочлен имеет корень $x_0$. Разделим его на
$(x-x_0)$. Получим новый многочлен. У него также есть корень 
(и далее по индукции). $\square$\\
\textbf{Следствие 2.} Пусть есть два многолчена степени n, и дан набор точек
таких, что $f(x_i)=g(x_i)$. Тогда нам требуется не менее n+1 таких точек, 
чтобы установить равенство многочленов. \\
\textbf{Доказательство}. Пусть $f(x_i)=g(x_i)$ для $i\in\{1,...n+1\}$. 
Рассмотрим $f(x)-g(x)$. Тогда все $x_i$ - корни этого многочлена, он имеет не
менее n+1 корней. Но его степень не превышает n. Значит, многочлен $f(x)-g(x)$ 
тождественно равен 0. Следовательно, по n+1 точке можно восстановить многочлен.
$\square$\\
\textbf{Задача}. Восстановить многочлен по набору х и f(х) мощности n+1. 
Для первой степени:
$$f=f(x_1)\frac{x-x_2}{x_1-x_2}+f(x_2)\frac{x-x_1}{x_2-x_1}$$
Для второй степени: $$f(x)=f(x_1)\frac{(x-x_2)(x-x_3)}{(x_1-x_2)(x_1-x_3)}+f(x_2)
\frac{(x-x_1)(x-x_3)}{(x_2-x_1)(x_2-x_3)}+f(x_3)\frac{(x-x_1)(x-x_2)}{(x_3-x_1)
(x_3-x_2)}$$
Общая формула - \textbf{интерполяционная формула Лагранжа}:
$$f(x)=\sum\limits_{i=1}^{n+1}f(x_i) \prod\limits_{j\ne i}\frac{(x-x_j)}
{(x_i-x_j)}$$
Доказательство - из второго следствия. \\
\textbf{Интерполяция} - построение многочлена по заданным точкам\\
\textbf{Экстраполяция} - продлевание занчений за область доступного нам
определения.\\
\textbf{Аппрокимация} - приближение набора данных некторым многочленом. \\
\textbf{Вопрос} На сколько корней можно разложить многочлен над
действительными числами?
Пусть $f(x)$ - многочлен над полем комплексных чисел, и
$x\in\mathbb C\backslash\mathbb R$ - его корень. Тогда $\overline{x}$ - 
также корень, поскольку $\overline{x^n}=\overline{x}^n$. Отсюда следует, 
что количество комплесных корней с ненулевой мнимой частью всегда четно, 
поэтому у многочленов нечетной степени хотя бы один корень действительный.\\
\begin{defin}
Неприводимый многочлен - многочлен, который нельзя представить в виде 
произведения многочленов, степень каждого из которой больше 0. 
\end{defin}
В $\mathbb R[x]$ все многочлены первой степени и некоторые многочлены второй
степени неприводимы.\\
В $\mathbb C[x]$ многочлен неприводим тогда и только тогда, когда его степень
равна 1.\\
\textbf{Пример}. $f=x^3-2x^2-13x-10\in\mathbb R[x]$. Разложите его на
неприводимые многочлены. Вообще говоря, как подбирать корни? По обобщенной
теоереме Виета. Свободный член - произведение корней на (-1) в степени четности 
степени многочлена, коэффициент при $x^n-1$ - минус сумма корней, и 
так далее(вставить для произвольного многчлена)\\
\textbf{Пример}. Найти корни в $f(x)=x^4-6x^2+7x-6$\\
\textbf{Задача}. Найти все неприводимые многочлены степени 3 в $\mathbb Z_2[x]$.
Ответ: $x^3+x+1$ и $x^3+x^2+1$.\\
\textbf{Задача}. Разложить $x^4+1\in\mathbb Z_3$. Корней нет, значит это 
произведение квадратных двучленов: $x^4+1=(x^2+2x+2)(x^2+x+2)$
\subsection{Дифференцирование многочленов}
Пусть $f(x)$ - многочлен. 
\begin{defin}
Производная многочлена $f(x)=a_nx^n+...+a_0$ есть многочлен 
$f'(x)=na_nx^{n-1}+...+a_1$
\end{defin}
Производная многочлена нулевой степени и многочлена-нуля по определению - ноль.\\
Вторая производная: $f''(x)=n(n-1)a_nx^{n-2}+...+a_2$\\
Доказать по опр: $f^{(n)}(x)=n!a_n$\\
\textbf{Свойтсва дифференцирования многочленов}\\
1. $(f+g)'=f'+g'$\\
2. $(fg)'=f'g+fg'$\\
Доказательство - по определению производной для произведения. \\
3. $f(g(x))'=f'(g)g'(x)$ - алсо по опред\\
4. $(f^m(x))'=mf^{m-1}(x)f(x)$\\
5. $x_0$ - корень кратности k многочлена $f(x)$ тогда и только тогда, когда
$x_0$ - корень кратности k-1 в $f'(x)$ (рассмотреть два случая: к=1 и
к больше 1, по свойствам 2 и 3)


\section{Рациональные дроби}
Рассмотрим кольцо многочленов $\mathbb F[x]$ над полем $\mathbb F$. 
Рассмотрим множество $$Q[x]=\left\{\frac{f(x)}{g(x)}\mid f,g\in\mathbb F[x],
g\ne0\right\}$$
Определим сложение и умножение как для обычных дробей. Мы видим, что сложение
и умножение многочленов находятся в этом множестве. Докажем, что это
множество образует поле: \\
1. Это абелева группа по сложению: ассоциативно, наличие нуля
(0: $f/g+0=(f+0g)/g=f/g$), наличие обратного (со знаком минус:
$f/g+(-f/g)=(f-f)/g=0$),
коммутативность (следует из комутативности сложения и умножения многочленов).\\
$f/g=h/k \Leftrightarrow fk=gh$. Доказательство: домножим на $gk$\\
$f/g=fh/gh$ - доказать по определению. \
2. Ассоциативность и дистрибутивность умножения: есть, есть \\
3. Коммутативность умножения - есть\\
4. Нейтральный элемент по умножению - 1\\
5. Делители нуля: нет\\
Следовательно, коммутативное кольцо с единицей без делителей нуля. Есть ли 
обратный элемент? Да: $(f/g)^{-1}=g/f$\\
Cледовательно, $Q[x]$ - поле рациональных дробей. В частнсти,
$\mathbb F[x]\subset Q[x]$
\subsection{Степень дроби}
\begin{defin}
$deg(f/g)=deg(f)-deg(g)$
\end{defin}
\begin{defin}
Дробь \textbf{правильная}, если $deg(f/g)<0$
\end{defin}
В частности, 0 - правильная дробь
\begin{defin}
Дробь несократимая, если НОД(f,g)=1
\end{defin}
\begin{theor}
Любую рациональную дробь можно представить в виде суммы многочлена и
правильной дроби
\end{theor}
\textbf{Доказательство.} Пусть $f/g$ - искомая рациональная дробь. 
Если $deg(f)<deg(f)$, дробь уже правильная (суммируем с нулевым многочленом).
Если $deg(f)\geqslant deg(f)$, произведем деление, получим $f=hg+q$, 
где q - остаток, степень которого меньше степени g. Значит, в равенстве
$f/g=h+q/g$ h - многочлен, q/g - правильная дробь. $\square$\\

Если $f/g=f/p^n$, p - неприводимый многочлен и $deg(f)<deg(p)$, то $f/g$ - 
\textbf{простейшая} дробь.\\
Например, над полем $\mathbb R$ неприводимы $\frac{A}{(x-k)^n}$
\begin{theor}
Любая правильная дробь представима в виде суммы простейших дробей.
\end{theor}
\textbf{Доказательство}.
Случай 1. Знаменатель раскладывается на произведение неприводимых дробей:
$f/(g_1g_2)=f_1/g_1+f_2/g_2$, $deg(f_i)<deg(g_i)$\\
Случай 2. $f/h=f/g^k=f_1/g+f_2/g^2+...f_k/g^k$, $deg(f_i)<deg(g)$ $\square$\\

\textbf{Пример}. Представить через простейшие 
$\frac{x^3-2x^2+3x+34}{x^4+2x^3-7x^2-8x+12}=\frac{x^3-2x^2+3x+34}{(x-1)
(x+3)(x-2)(x+2)}
=\frac{A}{x-1}+\frac{B}{x+3}+\frac{C}{x-2}+\frac{D}{x+2}$. 
Отсюда, если решать систему $A=-3,\quad B=1,\quad C=2,\quad D=1$. 
Но это - метод неопределенных коэффициентов. Рассмотрим 
\textbf{метод частных значений}: подставим туда удобные числа, например при х=1,
значение верхней дроби 36, откуда А=-3. При x=-2, получаем D=1, и так далее. \\
\textbf{Пример}. Разложить на простейшие над $\mathbb R$
$$\frac{x^2+x-2}{x^3-3x^2+4x-12}=\frac{x^2+x-2}{(x-3)(x^2+4)}=\frac{A}{(x-3)}+
\frac{Bx+C}{x^2+4}$$
Отсюда $x^2+2x+2=A(x^2+4)+(Bx+C)(x-3)$\\
Разложим по МЧЗ: при х=3, A=1\\
При х=0 и х=1, В= и С=2\\
\textbf{Пример}. Разложим над $\mathbb Z_2$
$$\frac{1}{x(x^3+x^2+x+1)}=\frac{1}{x(x+1)^3}$$
Ответ: $\frac{-1}{x}+\frac{1}{x+1}+\frac{1}{(x+1)^2}+\frac{1}{(x+1)^3}$\\
\begin{defin}
Пусть $f/g\in Q[x]$. Тогда его производная $(f/g)'=\frac{f'g-fg'}{g^2}$
\end{defin}

