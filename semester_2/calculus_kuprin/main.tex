\documentclass[a4paper]{article}

\usepackage[left=2.5cm,right=2.5cm,top=2.5cm,bottom=3cm]{geometry}
\usepackage[12pt]{extsizes}

\usepackage{hyperref}
\usepackage{xcolor}
\definecolor{linkcolor}{HTML}{799B03} % цвет ссылок
\definecolor{urlcolor}{HTML}{799B03} % цвет гиперссылок

\usepackage[russian]{babel}
\usepackage{amsmath}
\usepackage{amssymb}
\usepackage{amsfonts}


\title{Лекции по анализу}
\date{Семестр 2}
\author{Куприн А.В/Максимов Д.А}

\newtheorem{defin}{Определение}
\newtheorem{example}{Пример}
\newtheorem{zam}{Замечание}
\newtheorem{theor}{Теорема}

\begin{document}
\maketitle
\tableofcontents
\newpage
\section{Формулы всякие}
Площадь эллипса$=\pi ab$

\section{Площадь фигуры}
Пусть функция $f(x)$ неотрицательна и непрерывна на $[a,b]$. 
Тогда площадь под её графиком равна 
\begin{equation} \label{int}
  S=\int\limits^b_af(x)dx  
\end{equation}
Если функция задана параметрически $$f(t)=\begin{cases}x=x(t)\\y=y(t)
\end{cases}$$
То площадь под её графиком равна
\begin{equation} \label{intp}
    S=\int\limits^{t_2}_{t_1}y(t)x'(t)dt
\end{equation}
при условии $y(t)\geqslant0$, $x'(t)\geqslant0$\\
Если в параметрически заданном графике имеется петля (в которой $f(t_1)=f(t_2)$ 
- точка самопересечения), то площадь внутри этой петли также выражается
формулой \ref{intp}. Принимается, что если мы обходим петлю по часовой 
стрелке, то ориентированная площадь отрицательна, если против часовой стрелки



- то положительна. \\
Чтобы рассчитать площадь в полярной системе координат, вспомним, что по 
теоереме синусов, площадь треугольника равна $0.5AB\sin\varphi$. В пределе при
$d\varphi\to0$, $A=B=r(\varphi)$, а $\sin\varphi\sim\varphi$. Отсюда получаем
формулу
$$S=\frac{1}{2}\int\limits^{\varphi_2}_{\varphi_1}r^2(\varphi)d\varphi$$

Заметим, что если функция зависит также от некоторого размерного параметра, 
то искомая площадь зависит от его квадрата.  

\section{Длина линии}
По теореме Пифагора, дифференциал длины линии имеет вид
$dl^2=dx^2+dy^2$. Тогда в случае, если длина дуги стремится к длине хорды, 
длина линии имеет вид
$$L=\int\limits^b_adl$$
Поскольку $dl^2=dx^2(1+(dy/dx)^2)$, то длина графика функции имеет вид
$$L=\int\limits^b_a\sqrt{1+(f'(x))^2}|dx|,~a>b$$
Если на отрезке $[a,b]$ функция положительна, то можно не ставить модуль.\\
Для параметрически заданной функции имеем $dl^2=(x'_tdt)^2+(y'_tdt)^2$, 
поэтому
$$L=\int\limits^{t_2}_{t_1}\sqrt{x'^2_t+y'^2_t}dt$$
В данном случае модуль не требуется.\\
В полярных координатах функция $r=r(\varphi)$ переводится в декартовы как
параметрическая функция от $\varphi$: 
$$f(t)=\begin{cases}x=r(\varphi)\cos\varphi\\y=r(\varphi)\sin\varphi
\end{cases}$$
Тогда $dl^2=(d(r(\varphi)\cos\varphi))^2+(d(r(\varphi)\sin\varphi))^2$;
$(x'_\varphi)^2=(r'_\varphi\cos\varphi-r\sin\varphi)^2$, $(y'_\varphi)^2=
(r'_\varphi\sin\varphi+r\cos\varphi)^2$. Отсюда, раскрывая скобки, получаем 
длину кривой  
$$L=\int\limits_{\varphi_1}^{\varphi_2}\sqrt{r'^2_\varphi+r^2}d\varphi~~~
d\varphi\geqslant0,~\varphi_1<\varphi_2$$
\section{Объемы и площади поверхностей}
Объем как интегрирвоание поперечных сечений. Выделим ось X, которая "пронзает"
тело, причем его "крышки" перпендикулярны ей и имеют координаты a и b. Тогда
$$V=\int\limits^b_aS(x)dx$$
В частности, если это тело вращения вокруг оси, и его граница выражается 
непрерывной функцией $f(x)$, то $S(f(x))=\pi f^2(x)$ и соответственно 
$$V_x=\pi\int\limits^b_af^2(x)dx$$
Так как функция стоит в квадрате, то не надо бояться разных знаков!
Если мы хотим вращать вокруг оси Y и нелзя выразить функцию $x=x(y)$ через
игрек, тогда
$$V_y=V_\text{цилиндра}+\int\text{объемы цилиндрических слоев}$$
Конкретно, пусть $f(a)=c$, $f(b)=g$, и пусть $a\ne0$. Тогда
$$V_y=\pi a^2|d-c|+2\pi\int\limits^b_axf(x)dx=\pi a^2|d-c|+2\pi\int\limits^
b_ax(D-f(x))dx$$
\begin{theor}
(Паппа-Гульдена)\\
Если есть некоторая ось и фигура, не пересекающая эту ось, то объем тела
вращения фигуры относительно оси равен произведению площади фигуры на длину
окружности, которую при вращении описывает центр тяжести фигуры.
\end{theor}
\textbf{Пример.} Тело вращения полукруга - шар. Поэтому его объем $V=2\pi 
r_c\times \pi r^2/2=4\pi r^3/3$. Отсюда $r_c=4r/3\pi$

\section{Площадь поверхности вращения}
Пусть дана ось L и ограниченная кривая l, лежащая с ней в одной плоскости.
Тогда мы можем провращать l вокруг L, и мы получим поверхность вращения. 
Пусть $ds$ - дифференциал длины кривой, $R$ - расстояние до оси. Тогда 
дифференциал площади вращения равен $dS=2\pi Rds$, и если мы выбрали ось Х, то
$$S=2\pi\int\limits^b_a Rds=2\pi\int\limits^b_af(x)\sqrt{1+f'^2(x)}dx$$



\section{Несобственный интеграл}
Вспомним, что для интегрируемости необходима ограниченность функции. Также 
параметр разбиения интегральной суммы стремится к нулю. В несобственном 
интеграле пределы стремятся к бесконечности, или же на отрезке интегрирования 
есть разрыв, стремление к бесконечности. В таком случае нет места интегралу
Римана. Как это можно формально определить?  
\subsection{Несобственный интеграл I-го рода}
- по бесконечным промежуткам. \\
Пусть $f(x)$ - функция, интегрируемая по Риману на любом отрезке
действительных чисел. По формуле Ньютона-Лейбница, имеем
$$\int\limits^{+\infty}_{-\infty}f(t)dt:=\lim_{a\to\infty,b\to-\infty}
\int\limits^a_bf(t)dt=\lim_{a\to\infty}F(a)-\lim_{b\to-\infty}F(b)$$
Если предел существует, интеграл сходится, если предел равен бесконечности -
интеграл расходится. \\
Пределы к плюс и минус бесконечности, очевидно, вычисляются отдельно. Иногда 
могут получаться кракозябры типа $[\infty-\infty]$, так как пределы независимы,
то в таком случае интеграл расходится.  \\
Если пределы зависимы, то имеем случай V.P. - главное значение - от минус до
плюс бесконечности при том, что предел интегрирования один, пределы зависимы,
то есть $\lim\limits_{a\to\infty}\int\limits_{-a}^{a}$, но не
$\lim\limits_{a\to\infty}\int\limits_{-2a}^{a}$\\
Вопрос: как узнать, сходится ли тот или иной интеграл?

\textbf{Интегралы от неотрицательных функций}\\
(точнее, функции, не меняющие знак)\\

$$\int\limits^{+\infty}_{-\infty}f(t)dt=\lim_{a\to-\infty}
\int\limits^a_cf(t)dt+\lim_{b\to\infty}\int\limits^c_bf(t)dt$$



\subsection{Несобственный интеграл II-го рода} 
- от неограниченных функций. 
\begin{defin}
Интеграл II-го рода
\end{defin}


\end{document}
